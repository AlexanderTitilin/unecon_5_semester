\documentclass[14pt]{extarticle}
\usepackage{fontspec}
\usepackage[russian, english]{babel}
\setmainfont{Times New Roman}
\usepackage{amssymb}
\usepackage{setspace}
\onehalfspacing
\usepackage{amsmath}
\usepackage{listings}
\usepackage{indentfirst}
\setlength{\parindent}{1.25cm}
\usepackage[right=10mm,left=30mm,top=20mm,bottom=20mm]{geometry}
\newtheorem{theorem}{Теорема}
\newtheorem{definition}{Определение}
\newtheorem{corollary}{Следствие}[theorem]
\newtheorem{lemma}[theorem]{Лемма}
\title{}
\author{}
\date{}
\DeclareMathOperator{\pv}{PV}
\DeclareMathOperator{\cf}{CF}
\begin{document}
	\maketitle
	Изучаем системный анализ финансового рынка.

	Учебник. Асват Инвестиционная оценка активов
	\section{Системный Анализ Финансового рынка}
	\subsection{Рыночный риск}
	\begin{enumerate}
		\item Валютный
		\item Процентный
		\item Фондовый 
	\end{enumerate}
	Рыночные рынки управляются хеджированием.

	В теории инвестиций
	существуют ряд 
	моделей для измерения
	риска доходности
	\begin{equation}
		\pv = \sum_{i = 1}^{n} \frac{\cf_{i}}{(1 + r_{i})^{n}}
	\end{equation} 
	\begin{enumerate}
		\item $n$ срок жизни финансового актива
		\item $\cf_{i}$ cash flow (денежный поток) в момент времени $i$
		\item  $r_{i}$ ставка дисконтирования
	\end{enumerate}
\end{document}

