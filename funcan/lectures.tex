\documentclass[14pt]{extarticle}
\usepackage{fontspec}
\usepackage[russian, english]{babel}
\setmainfont{Times New Roman}
\usepackage{amssymb}
\usepackage{setspace}
\onehalfspacing
\usepackage{amsmath}
\usepackage{amsthm}
\usepackage{listings}
\usepackage{indentfirst}
\setlength{\parindent}{1.25cm}
\usepackage[right=10mm,left=30mm,top=20mm,bottom=20mm]{geometry}
\newtheorem{theorem}{Теорема}
\newtheorem{definition}{Определение}
\newtheorem{pred}{Предложение}[section]
\newtheorem{example}{Пример}[definition]
\newtheorem{corollary}{Следствие}[theorem]
\newtheorem{lemma}[theorem]{Лемма}
\DeclareMathOperator{\intt}{Int}
\newcommand{\bm}[1]{ \left.#1\right|}
\title{}
\author{}
\date{}
\begin{document}
\maketitle
\section{Учебники}
\begin{enumerate}
	\item Колмогоров
	\item Люстерник, Соболев (Краткий Курс Функционального Анализа)
	\item Вайнберг Функциональный Анализ
	\item Бахарев
\end{enumerate}
\section{Метрические пространства}
Пусть есть некоторое множество $M$,
мы хотим ввести предел (непрерывность, производную  и тд) на этом множестве.

Надо ввести расстояние (метрику).
\begin{definition}[Метрика]
	Метрикой $\rho$ на множестве  $M$ 
	называется отображение  $\rho : M \times M \to [0,+ \infty)$
	удовлетворяющее следущим свойствам (аксиомам):
	 \begin{enumerate}
		\item $\rho(x,y) \ge 0, \rho(x,u) = 0 \iff x = y$
		\item $\rho(x,y) = \rho(y,x)$
		\item  $\rho(x,y) \le \rho(x,y) + \rho(y,z)$
	\end{enumerate}
	Пара $(M,\rho)$ называется метрическим пространством.
\end{definition}
\begin{example}
	\begin{equation}
		M = \mathbb{R}, \rho(x,y) = |x- y|
	\end{equation}
\end{example}
\begin{example}
	\begin{equation}
		M = \mathbb{R}^{n}, ||x|| =  \sqrt{\sum_{i=1}^{n}x_{i}^2} 
	\end{equation}
\end{example}
\begin{example}[Транспортная метрика (Матхэтеннская)]
	\begin{equation}
		\rho(A,B) = \min \text{~ ломанная соединяющая } A, B
	\end{equation}
\end{example}
\begin{example}
	$M$ -- город
	 \begin{equation}
		 \rho(A,B) = \min \text{ время за которое можно добраться~} A \to B
	\end{equation}
\end{example}
\begin{example}
	$M$ -- множество всех непрерывных функций $f(t): [0,1] \to \mathbb{R}$ , $M = C([0,1])$ 
	\begin{equation}
		\rho(f_1,f_2) = \max_{t \in[0,1]} |f_1(t) - f_2(t)|
	\end{equation}
	$\max \exists $ по теореме Вейрштрасса
	\begin{equation}
		(M,\rho) = C[0,1]
	\end{equation}
	Это одно из \textbf{важнейших} пространств функционального анализа
\end{example}
\begin{example}
	Обозначим $M =  \{\text{Множество всех последовательностей}  \{x_{n}\} = (x_1,x_2,\dots,x_{n},\dots) , x_{k} \in \mathbb{R}\}$
	\begin{equation}
		\rho(\{x_{n}\},\{y_{n}\}) = 
		\sum_{k=1}^{\infty} \frac{1}{2^{k}} \frac{|x_{k} - y_{k}|}{1 + |x_{k} - y_{k}|}
	\end{equation}
	\begin{enumerate}
		\item Ряд сходится для любых последовательностей
			так как мажорируется рядом $\sum_{k=1}^{\infty} \frac{1}{2^{k}} = 1$
		\item Докажем, что выполняется неравентство
			треугольника
			
			Рассмотрим вспогательную функцию
			\begin{equation}
				f(t) = \frac{t}{1  +t} : [0,+\infty] \to \mathbb{R}
			\end{equation}

			Ясно что $f(t) = 1 -\frac{1}{1+t}$ 
			данная функция возрастаетс так как  $\frac{1}{1+t}$ убывает. Отсюда следует, что
			\begin{equation}
				\frac{|a+b|}{1 +|a+b|} 
				\le \frac{|a| + |b|}{1 + |a| + |b|} =
				\frac{|a|}{1+|a|+|b|} + \frac{|b|}{1 +|a| + |b|}\le  \frac{|a|}{1+|a|} +\frac{|b|}{1 + |b|}
			\end{equation}
			\begin{equation}
				f(|a+b|) \le  f(|a| +|b|) \le  f(|a|) + f(|b|)
			\end{equation}
			Мы доказали неравенство треугольника для
			всех членов ряда

			Рассмотим $\{z\}$

			 \begin{equation}
				\frac{|x_{n} - y_{n}|}{1+|x_{n} - y_{n}|} = \frac{|x_{n} - z_{n} + z_{n}  -y_{n}|}{1 + |x_{n} - z_{n} + z_{n} - y_{n}|}\le 
				\frac{|x_{n} - z_{n}|}{1+|x_{n}-z_{n}|} + \frac{|z_{n} - y_{n}|}{1+|z_{n} -y_{n}|}
			\end{equation}
			\begin{equation}
				\rho(x_{n},y_{n}) \le 
				\rho(x_{n},z_{n}) + \rho(z_{n},y_{n})
			\end{equation}
	\end{enumerate}
\end{example}
Понятие метрики позволяет на метрическом пространстве $(M,\rho)$
вводить <<старые>> понятия из анализа.
 \begin{enumerate}
	 \item Открытый шар радиуса $r$ с центром в точке $x_0$
		 \begin{equation}
		 	B_{r}(x_0) := 
			\{x \in M \mid \rho(x,x_0) < r\}
		 \end{equation}
	
	 \item Замкнутый шар радиуса $r$ с центром в точке $x_0$
		 \begin{equation}
		 	\overline{B}_{r}(x_0) := 
			\{x \in M \mid \rho(x,x_0) \le  r\}
		 \end{equation}
	\item $X \subset M$ назывется открытым, если  $\forall x \in X ~\exists B_{r}(x) \subset X$ 
	\item Множество $X$ называется замкнутым если дополнение к нему ($M \setminus X$) является открытым
	\item Точка $x_0$ называется внутренней точкой $X$, если $\exists  B_{r}(x_{0} \subset X$ ,  $X$ открытое  $\iff$ любая точка внутренняя
	\item $x_0$ называется предельной точкой множества $X$,
		если  $\forall  r > 0$ $B_{r}(x_0) \cap X$ содержит
		бесконечно много точек из $X$
        \item $x_0$ называется изолированной точкой множества
		$X$  если  $\exists  B_{r}(x_0): B_{r}(x_0) \cap X = \{x_0\}$
	Изолированная точка не может быть предельной
	\item Точка $x_0$ называется внешней для множетсва $X$,
		если существует такой шар с центром в  $x_0$,
		что его пересечение с $X$ пусто
	\item Точка $x_0$ называется граничной точкой множества $X$ 
		если  $\forall r$ в шаре $B_{r}(x_0)$ содержатся
		точки как $x \in X$, так и $x \notin X$
\end{enumerate}
Коллекция фактов (без доказательсва, упражнения, дз)
\begin{enumerate}
	\item Другое определение замкнутоcти.  $X$ замкнуто  $\iff$ содержит все свои предельные точки
	\item Добавление к  $X$ всех его предельных точек 
		называется пополнением  $X$. Полученное множество
		обозначат  $\overline{X}$
		 \begin{equation}
			 \overline{X} = X \cup \{\text{~Пределные точки~}\}
		 \end{equation}
	\item $\overline{X}$ замкнутое
	\item $X$ замкнутое  $\iff$  $X = \overline{X}$
	 \item Принцим трихотомии (деления на 3)
		 $\forall $ множества $X$ и  $\forall  x \in M$ 
		 возможен только один из трех вариантов
		 \begin{enumerate}
		 	\item $x$ внутренняя точка $x \in \intt X$
			\item  $x$ граничная точка $x \in \delta X$
			\item  $x$ внешняя точка
		 \end{enumerate}
		Верны формулы
		\begin{enumerate}
			\item $\overline{X} = X \cup \delta X$
			 \item $\overline{X_1 \cup X_2} = \overline{X_1} \cup \overline{X_2}$
		\item Объединение любого числа открытых множеств открыто
		\item Пересечение конечного числа открытых множеств открыто
		\item Пересечение любого числа замкнутых множеств
			замкнуто
		\item объединение конечного числа замкунтых множеств
		\item объедение бесконечного числа замкнутых множеств может быть открытым
		\end{enumerate}
\end{enumerate}
\section{}
Понятие метрики позволяет определять на $M$ 
понятия сходимости и непрерывности.

Рассмотрим $\{x_{n}\} \in M$
\begin{definition}
	$x_{n} \to a \in M$  если 
	\begin{equation}
	\forall  \epsilon >  0 \exists  N = N(\epsilon) : \forall   n \ge  N~\rho(x_{n},a) < \epsilon
	\end{equation} 
\end{definition}
Сохраняются многие свойства обычного
предела $x_{n} \to a \iff \rho(x_{n},a) \to 0 ,n \to \infty$
\begin{pred}
	$\exists  \lim$ то он единственный
\end{pred}
\begin{proof}
	Пусть $x_{n} \to a_1$ , $x_{n} \to a_2$ , $a_1 \neq a_2$
	\begin{equation}
	\rho(a_1,a_2) \le \rho(a_1,x_{n} ) + \rho(x_{n},a_{a_2})
	\end{equation} 
	\begin{equation}
	\rho(a_1,x_{n}) \to 0
	\end{equation} 
	\begin{equation}
	\rho(a_2,x_{n}) \to 0
	\end{equation} 

\begin{equation}
0 \le  \rho(a_1,a_2) \le  0
\end{equation} 
\begin{equation}
\rho(a_1,a_2) = 0 \implies a_1 = a_2
\end{equation} 
\end{proof}
Теперь надо ввести понятие непреывности. 
Пусть $(M_1,\rho_1)$, $(M_2,\rho_2)$ два метрических пространства

Рассмотрим $f : M_1 \to M_2$
\begin{definition}
	$f$ непрерывно в точке  $x_0 \in M_1$ если
	$\forall  \epsilon > 0 \exists  \delta = \delta(\epsilon) :
	\forall x \rho_1(x,x_0) < \delta : \rho_2(f(x),f(x_0)) < \epsilon$
\end{definition}
\begin{definition}
	 $f$ непрерывна на  $M_1$ $\iff$ она непрерывна в любой точке
	  $M_1$
\end{definition}
\begin{pred}
	Метрика $\rho$ автоматически непрерывная функция
	 $\rho: M \times M \to \mathbb{R}$
\end{pred}
\begin{proof}
	$\forall  (x_0,y_0) \in M \times M$ 
	\begin{equation}
	\forall  \epsilon > 0 \exists  \delta = \delta(\epsilon) :
	\forall x: \rho(x,x_o) < \delta,
	\forall  y  \rho(y,y_0) < \delta 
	\implies 
	\overline{\rho} ((x,y),(x_0,y_0)) < \epsilon
	\end{equation} 
	Пусть $\overline{\rho}((x,y),(x_0,y_0)) = \rho(x,x_0)+\rho(y,y_0)$
	\begin{equation}
	\delta = \frac{\epsilon}{2}
	\end{equation} 
	\begin{equation}
	\overline{\rho((x,y),(x_0,y_0))} < \frac{\epsilon}{2} +
	\frac{\epsilon}{2} = \epsilon
	\end{equation} 
\end{proof}
\begin{example}
	$\exists $ метрика в которой любая функция непрерывна
	\begin{equation}
	\rho(x,y) = 
	\begin{cases}
		0, x= y\\
		1, x\neq y
	\end{cases}
	\end{equation} 
\end{example}
\begin{definition}
	$\{x_{n}\}$ называется фундаментальной (или последовательностью Коши) или сходящейся в себе, если 
	\begin{equation}
	\forall  \epsilon > 0 \exists  N=N(\epsilon) :
	\forall n , m \ge  N : \rho(x_{n},x_{m}) < \epsilon
	\end{equation} 
	Другой вариант 
	\begin{equation}
	\forall  \epsilon  > 0, \exists  N = N(\epsilon) :
	\forall n \ge  N , \forall m \in \mathbb{N}:
	\rho(x_{n},x_{n + m}) < \epsilon
	\end{equation} 
\end{definition}
\begin{pred}
	В любом метрическом постранстве, если
	последовательность сходится, то она фундаментальная
\end{pred}
\begin{proof}
	Пусть $x_{n} \to a \implies$
	 \begin{equation}
	 \forall  \epsilon \exists  N: \forall n \ge  N, 
	 \rho(x_{n},a) < \frac{\epsilon}{2}
	 \end{equation} 
	 \begin{equation}
	 n,m \ge  N 
	 \end{equation} 
	 \begin{equation}
	 \rho(x_{n},x_{m}) \le \rho(x_{n},a) + \rho(x_{n},a) <
	 \frac{\epsilon}{2} + \frac{\epsilon}{2}  = \epsilon
	 \end{equation} 
\end{proof}
\begin{definition}
	Метрическое пространство $(M,\rho)$ 
	если  $\forall $ фундаментальная последовательность
	имеет предел 
	(из фундаментальность  следует сходимость)
\end{definition}
\begin{theorem}
$\forall $ метрического пространства
$(M,\rho)$  $\exists $ полное метрическое пространство
$\overline{M},\overline{\rho}$
 \begin{equation}
	 M \subset \overline{M},\bm{\overline{\rho}}_{M} = \rho
 \end{equation} 
\end{theorem}
\begin{example}
	Рассмотрим $\mathbb{Q}$  $\rho(r_1,r_2) = |r_1 - r_2|$
	Оно не полное

	\begin{equation}
		\{r_{n }\} \to \sqrt{2} \notin  \mathbb{Q}
	\end{equation} 
	$\{r_{n}\}$ фундаментальная последовательность в  $\mathbb{R}$, значит она фундаментальная последовательность в $\mathbb{Q}$
\end{example}
\section{Пространство Чебышева}
Рассмотрим множество непрерывных
функций $f[a;b] \to \mathbb{R}$ $C([a,b])$
\begin{definition}[метрика Чебышева]
	\begin{equation}
		\rho_{C}(f_1,f_2) = \max_{x \in [a,b]} |f_1(x) - f_2(x)|
	\end{equation} 
\end{definition}
$\exists  \max$  по теореме Вейерштрасса
\begin{enumerate}
	\item $\rho_{C}$ очевидно , $\phi_{C}(f_1,f_2) = 0 \iff
		\max_{[a,b]}|f_1(x) - f_2(x)| = 0 \iff
		|f_1(x) - f_2(x)| =  0$
	\item очевидно
	\item 
		\begin{equation}
		\rho_{C}(f_1,f_2) \le 
		\rho_{C}(f_1,g) +
		\rho_{c}(g,g_2)
		\end{equation} 
		\begin{equation}
			\max_{[a,b]} |f_1(x) -f_2(x)|
			\le \max_{[a,b]}(|f_1 - f| + |g - f_2|)
			\le \max_{[a,b]} |f_1 - g| + \max_{[ a,b ]} |g-f_2|
		\end{equation} 
\end{enumerate}
Посмотрим, что означает сходимость по метреке  Чебышева
\begin{equation}
f_{n} \to f \iff \rho_{c}(f_{n},f) \to 0
\iff \max_{[a,b]} |f_{n}(x) - f(x)| \to 0
\end{equation} 
Это равномерная сходмость $f_{n}\rightarrow $
\begin{definition}
	$f_{n}(x)$ сходится $f(x)$ 
	поточечно $[a,b]$ если 
	$\forall x \in [a,b] $ , $f_{n}(x) \to f(f)$
\end{definition}
\begin{definition}
	$f_{n}(x) $ сходится к $f(x)$  равно мено на  $[a,b]$
	если  $\forall  \epsilon > 0 \exists  N = N(\epsilon) ~
	\forall n\ge  N$, то 
	$|f_{n}(x) - f(x)| < \epsilon \forall  x \in [a,b]$
	\begin{equation}
		\max_{[a,b]} |f_{n}(x) - f(x)| < \epsilon
	\end{equation} 
\end{definition}
$[a,b] = [0,1]$ , $f_{n}(x) = x^{n}$
 \begin{enumerate}
 	\item $x = 1, f_{n}(x) = 1$
	 \item $x \in [0,1)$   $f_{n}(x) = x^{n} \to 0$
 \end{enumerate}
 \begin{equation}
	 f_{n}(x) \to f(x) 
	 \begin{cases}
		 0 , x\in[0,1)\\
		 1 , x = 1
	 \end{cases}
 \end{equation} 
 $f \notin C$
  \begin{theorem}
 	Равномерный предел непрерывных функций
	есть непрерывная функция
 \end{theorem}
 \begin{theorem}
	 $C([a,b])$ с метрикой  $\rho_{c}$ полное
 \end{theorem}
 \begin{proof}
	 $\{f_{n}(x)\}$  фундаментальная последовательность.
	 \begin{equation}
		 \forall  \epsilon > 0 \forall  x\in [a,b] 
		 \exists  N = N(e,x) :
		 \forall n,m \ge  N \implies |f_{n}(x) -f_{m} (x) | <\epsilon
	 \end{equation} 
	 Так как $\mathbb{R}$ полно $\exists  \lim_{n \to \infty} 
	 f_{n}(x) = f(x)$
 \end{proof}
\end{document}


