\documentclass[14pt]{extarticle}
\usepackage{fontspec}
\usepackage[russian, english]{babel}
\setmainfont{Times New Roman}
\usepackage{amssymb}
\usepackage{setspace}
\usepackage{tikz}
\onehalfspacing
\usepackage{amsmath}
\usepackage{amsthm}
\usepackage{listings}
\usepackage{indentfirst}
\usepackage{graphicx}
\setlength{\parindent}{1.25cm}
\usepackage[right=10mm,left=30mm,top=20mm,bottom=20mm]{geometry}
\newtheorem{theorem}{Теорема}
\newtheorem{definition}{Определение}
\newtheorem{example}{Пример}
\newtheorem{corollary}{Следствие}[theorem]
\newtheorem{lemma}[theorem]{Лемма}
\DeclareMathOperator{\mrts}{MRTS}
\DeclareMathOperator{\mrs}{MRS}
\DeclareMathOperator{\wpp}{WP}
\title{Математическая экономика}
\author{}
\date{}
\begin{document}
	\maketitle
	\begin{enumerate}
		\item Дмитририев Антон Леонидович
		\item dmitr7171@mail.ru
		\item Элементы математической экономики, Экланд (как сказка на ночь)
		\item Аллен Р.Г Математическая экономия
		\item Тарасевич Микроэкономика
		\item Микроэкономика практикум ii (Дмитриев)
	\end{enumerate}
	\section{Предпочтения}
	Нам надо научиться математически
	моделировать индивида.
	\begin{definition}[Поведенческий Постулат]
		Лицо принимающее решение
		всегда выбирает наиболее предпочтительную
		для себя альтернативу
	\end{definition}
	Модель выбора должна содержать:
	\begin{enumerate}
		\item описание системы предпочтений ЛПР
		\item множество альтернатив, доступных ЛПР
	\end{enumerate}
	\begin{enumerate}
		\item В основе выбора
			наилучшей альтернативы лежит
			сравнение возможных вариантов
		\item сравнение любых
			альтернатив предполагает их прямое или
			косвенное сопоставление
		\item ЛПР сравнивают любую пару
			возможных вариантов по приницу
			лучше хуже. С точке срение математики 
			задается бинарное отношение.
	\end{enumerate}
		Пусть $М$
		множество непустое ЛПР
		альтернатив. Рассмотрим
		множество всех упорядоченных пар  $(x,y)$,
		 $(x,y) \sim (y,x)$ безразличие
	 \begin{definition}[Бинарное отношение]
		 \begin{equation} 
		 A \subseteq M \times M
		 \end{equation} 
	\end{definition}
	\begin{definition}[Функция]
		$\forall x \in M \exists ! y \in M$ для которого
		справедливо $x A y$
	\end{definition}
	\begin{definition}[График]
		\begin{equation} 
			\Gamma = \{(x,y) \mid x,y \in M, xAy\} 
		\end{equation} 
	\end{definition}
	\begin{definition}[Отношение предпочтения]
		Потребитель сравнивает два набора благ
		\begin{enumerate}
			\item строгое предпочтение ($x$ лучше  $y$)
			\item слабое предпочтение $x$ не хуже  $y$
			\item  безразличие,  $x$ и  $y$ одинаково хороши
		\end{enumerate}
	\end{definition}
	Рассмотрим символьную запись
	\begin{enumerate}
		\item $x \succ y$  $x$ строго лучше  $y$  
		\item $x \succeq y$ $x$ не хуже  $y$
		\item  $x \sim y$  $x,y$ одинакого предпочтительны
	\end{enumerate}
	\subsection{Гипотезы (аксиомы) о свойствах}
	\begin{enumerate}
		\item \textbf{ Полнота } для любых наборов выполняется $x \succeq y$ или $y \succeq x$
		\item \textbf{Рефлексивность} для любого $x$ ,  $x \succeq x$
		\item \textbf{Транзитивность},
			есть три набора благ
			$x,y,z$
			 \begin{equation} 
			x \succeq  y \land y \succeq z \implies x \succeq z
			\end{equation} 
	\end{enumerate}
	\subsection{Непрерывность отношения}
	\begin{definition}
		Отношение на множестве $X$ 
		непрерывно
		если для любого вектора
		 $y \in X$  
		 множества
		  \begin{equation} 
			  \{x \in X \mid x \succeq y\} 
		 \end{equation} 
		 \begin{equation} 
			 \{x \in X \mid x \preceq y\} 
		 \end{equation} 
		 являются замкнутыми
	\end{definition}
	\begin{definition}[Рациональное отнощение потребления]
		Определенное на множестве
		наборов благ $R^{n+}$ 
		отношения предпочтения $\succeq$ 
		называется
		рациональным, если оно является:
		\begin{enumerate}
			\item полным
			\item рефлексивным
			\item транзитивным
		\end{enumerate}
	\end{definition}
	\subsection{Свойства рационального отношения предпочтения}
	В случае рациональность $\succeq$ :
	\begin{enumerate}
		\item $\succ$ антирефлексвно  ( не выполняется $x \succ x$ ), транзитивно
		\item $\succeq$ рефлексивно,транзитивно,симметрично
		\item $x \succ y \succeq z \implies x \succ z$
	\end{enumerate}
	\subsection{Кривые безразличия}
	\begin{enumerate}
		\item Зафиксируем некоторый набор благ
			$x'$ .
			Множество всех набор одинаково
			предпочтительных с  $x'$,
			называется кривой безразличия, содержашей  $x$
	\item Посколько кривая не сегла кривая в геометрическом
		смысле, то правильно говорить о
		множестве безразличия
		\begin{equation} 
			I(x') = \{y \in R^{n+} \mid y \sim x'\} 
		\end{equation} 
	\end{enumerate}
	\begin{equation} 
	\wpp(x) 
	\end{equation} 
	множество наборов не хуже $x$,  $I(x) \subseteq \wpp(X)$

	Кривые безразличя не пересекаются, возникнет нарушение транситивности
	\subsection{Наклон кривых безразличия}
	\begin{enumerate}
		\item Товар наличие которого в большем количестве
			всегда предпочтительнее меньшего, называется
			\textbf{благом}
		\item Если в наборе присутствуют только блага,
			то кривая безразличия имеет отрицательный
			наклон, по отношению к 
			осям соответсвующих благ
	\end{enumerate}
	\begin{definition}[Антиблаго]
		Товар, наличие которого
		в наборе в меньшем количестве всегда предпочтительнее большего называется \textbf{антиблаго}
	\end{definition}
	\begin{definition}[Совершенные заменители]
		Если потребитель в любых
		условиях считает два блага эквивалетными,
		то он совершенные заменители.

		Если набор состояит из совершенных
		заменителей, то предпочтительность определяется
		общим количеством.
	\end{definition}
	\begin{definition}
		Если потрибитель во всех ситуациях
		исползует блага 1,2 в некоторой
		фиксированной пропорции,
		такие блага называются \textbf{совершенными дополняемыми}
		\begin{equation} 
			U = \min(x_1,x_2)
		\end{equation} 
	\end{definition}
	\begin{definition}
		Набор благ, строго предпочитаемый всем
		другим, называется точкой насыщения
	\end{definition}
	\begin{definition}
		Отношение предпочтения
		мы будем называть локально ненасышенным
		дл любого набора
		$x \subseteq R^{n+}$ 
		и произвольного числа $t > 0$ 
		найдется  $y \subseteq R^{n + }$ 
		$y - x \preceq t$ и при этом $y \succeq x $
	\end{definition}
	\begin{enumerate}
		\item Бесконечно делимое благо
		\item Дискретное благо
	\end{enumerate}
	\begin{definition}
		Рациональное отношение 
		предпочтения, является регулярным
		если оно
		\begin{enumerate}
			\item монотонно, большее 
				количество блага всегда предпочитается меньшему (наборы состоят только из благ)
			\item выпуклое
		\end{enumerate}
	\end{definition}
	Выпуклая комбинация двух различных, но при этом
	одинаково предпочтительных наборов предпочтительных или по крайней мере не хуже, чем каждый из сотавляющих наборов
	\subsection{Наклон кривых безразличия}
	Вычисленный в конкретной тчоке наклон,
	кривой безразличия характеризует в ней 
	предельную норму замены благ MRS (margina rate
	of substitution)

	$\mrs$ в точке  $x'$ 
	харакетризует исчисленный в ней наклон
	кривойй безразличия, которому эта точка принадлежитю
	Геометрически  $\mrs$ есть тангенс угла наклона касательной
	и кривой без различия в точке  $x'$

	$\mrs$ в точке  $x'$ есть  $\lim_{\Delta x_1 \to 0} \frac{\Delta x_2}{\Delta x_1}$ или иначе $\frac{d x_2}{d x_1}$ в точке $x'$


	Если набор составлен из двух благ, то соответсвующие кривые
	имеют отрицательный наклон  $\mrs < 0$

	Если набор включает одно благо и одно антиблаго $\mrs > 0$
	\section{Полезность}
	\begin{definition}
		Функция полезност $U(X)$ описывает отношение
		предпочтения  $\succeq$ тогда
		и только когда для двух наборов благ  $x', x''$ 
		верны следущие соотношения
		 \begin{enumerate}
			\item $x' \succ x'' \iff U(X') > U(x'')$
			\item  $x' \prec x'' \iff U(x') < U(x'')$
			\item  $x' \sim x'' \iff U(x') = U(x'')$ 
		\end{enumerate}
	\end{definition}
	Непрерывное, монотонное возрастающее рциональное
	отношение педпочтения может
	быть представлено непреывной функцией полезности

	\begin{definition}[непрерывность]
		Малые изменения набора благ, ведут малые изменения
		предпочтительности набора благ
	\end{definition}
	Полезность является порядковым (задающим упорядочение) 
	понятием
	\begin{example}
		$U(x) = 6,U(y)=2$ x строго предпочтительнее $y$,
		но при этом нельзя сказать, что $x$ в 3 раза
		предпочтительнее  $y$
	\end{example}
	\begin{example}
		Рассмотрим наборы благ, представленные векторами
		$(4,1),(2,3),(2,2)$
		
		Допустим 
		 \begin{equation}
			 (2,3) \succ (4,1) \sim (2,2)
		\end{equation} 
		Поставим в
		соотсветвие этим  наборам произвольные числа,
		сохранябщие упорядчение векторов по предпочтительности
		\begin{equation}
		U(2,3) = 6 > U(4,1) = U(2,2) = 4
		\end{equation} 
		Назовем эти числа уровнями полезности
	\end{example}
	\begin{definition}
		Совокупность всех кривых безразличия
		называется картой кривых безразличия
	\end{definition}
	\begin{example}
		Пусть $U(x_1,x_2) = x_1 x_2$ описывает
		отношение $\succeq$
		\begin{equation}
		V = U^2 = x_1^2 x_2 ^2
		\end{equation} 
		\begin{equation}
			V(2,3) = 35 > V(4,1) = V(2,2) = 16
		\end{equation}
		\begin{equation}
			(2,3) \succ (4,1) \sim (2,2)
		\end{equation} 
		$V$ сохраняет тоже самое упорядочение, что и
		 $U$. Функция описывает одинаковое с  $U$ отношение предпочтения
		  \begin{equation}
		 W = 2 U +10
		 \end{equation} 
		 \begin{equation}
			 W(2,3) = 22 > W(4,1) = W(2,2) = 18
		 \end{equation} 
	\end{example}
	Если $U(x)$ является функцией полезности
	описыващей отношение предпочтения  $\succeq$
	на множестве неотриательных наборов блаш  $R^{n+}$ ,
	$f(U)$ есть строго возрастающая функция, одного
	аргумента, то зависимость  $V = f(U)$ так же
	представляет собой функцию полезности,
	описывающую исходное отношение предпочтения  $\succeq$

	\begin{theorem}[о существовании непрерывной функции полезности]
	Пусть отношения предпочтения ЛПР $\succ$
	является полным, рефлексивным, непрерывным и строго монотонным.
	Тогда существует непрерывная функция полезности  $U:R^{n+} \to R$ описывающая данное отнощение предпочтения
	\end{theorem}
	Рассмотрим $V = x_1 + x_2$ ее кривые безразличия это
	прямые линии, состоящих из совершенных заменителей

	Рассмотрим $W(x_1,x_2) = \min(x_1,x_2$ ее кривые безразличия
 блага совершенные дополнители


Рассмотрим $U(x_1,x_2) = f(x_1) + x_2$ квазилинейная функция, является линейной только по $x_1$
\begin{definition}[Функция Кобба-Дугласа]
	\begin{equation}
		U(x_1,x_2) = x_1^{a} x_2^{b}, a>0,b>0
	\end{equation}
\end{definition}
\begin{definition}[Предельная Полезность]
	Предельная полезность продукта $i$
	\begin{equation}
		MU_{i} = \frac{\partial U}{\partial x_{i}}
	\end{equation} 
\end{definition}
\begin{example}
	\begin{equation}
	U(x_1,x_2) = x_1^{\frac{1}{2}}x^2_2
	\end{equation} 
	\begin{equation}
	MU_1 = \frac{1}{2} x_1 ^{- \frac{1}{2}}x^2_{2}
	\end{equation} 
	\begin{equation}
	MU_1 = 2 x_1^{\frac{1}{2}} x_2
	\end{equation} 
\end{example}
Общее уравнение кривой безразличия функции 
полезности $U(x_1,x_2)$ иметт вид $U(x_1,x_2) = k, k > 0 , k = \text{const}$. Полный дифференциал
\begin{equation}
\frac{\partial U}{\partial x_{1}} dx_1 + \frac{\partial U}{\partial x_2} dx_2 = 0
\end{equation} 
\begin{equation}
\frac{\partial U}{\partial x_2}dx_2
=-\frac{\partial U}{\partial x_1} dx_1
\end{equation} 
\begin{equation}
	\frac{d x_2}{dx_1}  = -\frac{\frac{\partial U}{\partial x_1}}{\frac{\partial U}{\partial x_2}}
\end{equation} 
Это $\mrs$
\begin{example}[квазилинейная функция]
	\begin{equation}
	U(x_1,x_2) = f(x_1) =x_2
	\end{equation} 
	\begin{equation}
	\frac{\partial U}{\partial x_1} = f'(x_1)
	\end{equation} 
	\begin{equation}
	\frac{\partial U}{\partial x_2} = 1
	\end{equation} 
	\begin{equation}
	\mrs = -\frac{\frac{\partial U}{\partial x_1}}{\frac{\partial U}{\partial x_2}} = - f'(x_1)
	\end{equation} 
	$\mrs$ не зависит от  $x_2$,
	наклон кривой безразличия постоянен во
	всех точках с фиксированным $x_1$
	(вдоль вертикальной линии, исходящей из точки ($x_1,0$))
\end{example}
\subsection{Монотонное преобразование функции
полезности и MRS}
Что происходит с предельной нормой замены при замене некоторой исходной функции ее монотонно возрастающим преобразованием.

\begin{enumerate}
	\item $U(x_1,x_2) = x_1 x_2 , \mrs = -\frac{x_2}{x_1}$
	\item $V = U^2$
	\item $V(x_1,x_2) = x_1^2 x_2^2$
		\begin{equation}
		\mrs = -\frac{x_2}{x_1}
		\end{equation} 
\end{enumerate}
В общем случае $V = f(U)$ , $f$ строго возрастает
 \begin{equation}
-\frac{\frac{\partial V}{\partial x_1}}{\frac{\partial V}{\partial x_2}} =- \frac{\frac{\partial U}{\partial x_1}}{\frac{\partial U}{\partial x_2}}
\end{equation} 
$\mrs$ инвариантно относительного любого
монотонно возрастающего преобразования функции полезности
\section{Выбор потребителя}
\subsection{Рациональный выор пр наличии ограничений}
\begin{enumerate}
	\item Наиболее предпочтительнй набор благ из числа доступных
		называется индвидуальным спросом
		при заданных ценах и доходе;
\end{enumerate}
В случае $x_1^{*} > 0, x_2^{*}>0$ 
то потребительский набор называется внутренним решением задачи
максмизаци полезности при наличии бюджетного ограничения

Если на приобретение
набора благ $(x_1^{*},x_2^{*})$ 
требуется $\$m$, то в этом случае расходуется весь доход

Наклон касательной и наклон бюджетной линии совпадают

Набор  $(x_1^{*},x_2^{*})$ 
представляет собойй решение экстремальной задачи,
описывающей потребительский выбор. Модель максимализации
полезность
\begin{equation}
\begin{cases}
	u(x_1,\dots,x_{n}) \to \max\\
	\sum_{i = 1}^{n} p_{i} x_{i} \le  m
\end{cases}
\end{equation} 
Если система представлена неоклассической потребность
$u(x)$ то решение с помощью метода лагранжа

 $u(x)$ описывает, рациональное, монотонное, выпуклое и
 непрерывное предпочтение

 функция полезность дважды непреывна дифференцируема
 \begin{equation}
	 u_{ij} (x) = \frac{\partial ^2 u(x)}{\partial x_{i} 
	 x_{j}}  i , j = 1 \dots n
 \end{equation} 
 \begin{equation}
 M U_{i}(x) = \frac{\partial u(x)}{\partial x_{i}} \ge 0
 \end{equation} 
 \begin{equation}
 MU (x) = (\frac{\partial u(x)}{\partial x_{i}})_{i= 1 \dots n} = 0
 \end{equation} 
 Функция полезности является матрица Гессе 
 \begin{equation}
 H^{u}(x) =  (\frac{\partial ^2 u(x)}{\partial x_{i} \partial x_{j} })
 \end{equation} 
 Отрицательно определена, в частности 
 $\frac{\partial ^2 u(x)}{\partial x_{i}^2} < 0$

 В точке оптимального выбора потребител $x^{*} = (x_1 ^*,
 \dots x_{n}^{*}$ 
 \begin{equation}
 MU _{i}(x^{*}) = \frac{\partial u(x^{*})}{\partial x_{i}} = \lambda p_{i}
 \end{equation} 
 \begin{equation}
 \frac{\partial u(x^{*})}{\partial x_{i}} :
 \frac{\partial u (x^{*})}{\partial x_{j}} = \frac{p_{i}}{p_{j}}
 \end{equation} 
 Предельная норма замены товара $j$ товаром  $i$
 \begin{equation}
 \mrs_{ij}(x^{*}) = 
 \frac{dx_{i}}{d x_{j}}(x^{*}) = 
 -\frac{\partial u(x^{*})}{\partial x_{j}} :
 \frac{\partial u(x^{*})}{\partial x_{i}} = -\frac{p_{j}}{p_{i}} i,j  = 1 \dots n
 \end{equation} 
 \begin{equation}
 \frac{\partial u(x^{*})}{\partial x_{j}}  : p_{j} =
 \frac{\partial u(x^{*})}{\partial x_{i}} :p_{i}
 \end{equation} 
 Для решения задачи ММП нужны необходимые условия Куна-Таккера

 Если набор благ $x^{*}(p,M)$ 
 являетя оптимальным, то существует множитель лагранжа
 $\lambda > 0$ , такой что
  \begin{equation}
	  \frac{\partial u(x^{*}(p,M))}{\partial x_{j}} \le  \lambda p_{j}
 \end{equation} 
 \subsection{Кобб-Дуглас}
 \begin{equation}
 U(x_1,x_2) = x_1^{a} x_2^{b}
 \end{equation} 
 \begin{equation}
 M U_{1} = \frac{\partial U }{\partial x_1} = a x_1^{a- 1} x_2^{b}
 \end{equation} 
 \begin{equation}
 MU_2 = \frac{\partial U}{\partial x_2} = bx_1^{a} x_2^{b-1}
 \end{equation} 
 \begin{equation}
 \mrs = - \frac{\frac{\partial U}{\partial x_1}}{\frac{\partial U}{\partial x_2}} = -1\frac{ax_2}{bx_2}
 \end{equation} 
 В точке $x^{*}$ $\mrs = - \frac{p_1}{p_2}$ 
 \begin{equation}
 - \frac{a x_{2}^{*}}{b x_{1}^{*}} = -\frac{p_1}{p_2}
 \end{equation} 
 \begin{equation}
 x^{*}_{2} = \frac{b p_1}{a p_2} x_2^{*}
 \end{equation} 
 \begin{equation}
 p_1 x_1^{*} + p_2 x^{*}_{2} = m
 \end{equation} 
 \begin{equation}
 p_1 x_1^{*} + p_2 \frac{b p_1}{a p_2} x^{1} =m
 \end{equation} 
 \begin{equation}
 x_1^{*} = \frac{am}{(a+b)p_1}
 \end{equation} 
 \begin{equation}
 x^{*}_{2} = \frac{bm}{(a + b) p_2}
 \end{equation} 
 \begin{equation}
	 (x_1^{*}, x_2^{*}) = 
	 \left(\frac{am}{(a+b) p_1},\frac{bm}{(a+b)p_2}\right)
 \end{equation} 
 \section{Рациональный выбор при наличии ограничений}
 Если одно из благ нулевое, то эта ситуация
 называется угловым решением
 задачи максимализации полезности при наличии
 бюджетного ограничени

 Если $U = x_1 + x_2$ , то наиболее предпочтительный набор
 \begin{equation}
	 (x_1^{*} ,x_2^{*}) = \left(\frac{m}{p_1},0\right), p_1<p_2
 \end{equation} 
 \begin{equation}
	 (x_1^{*},x_2^{*}) = \left(0,\frac{m}{p_2}\right), p_1 > p_2
 \end{equation} 
 При $p_1 = p_2$ все, расположенные на линии бюджетного
 ограничения наборы благ, оказыввются наиболее предпочтительными
 \begin{definition}[Функция Леонтьевского типа]
	 \begin{equation}
	 U(x_1,x_2) =  \min(ax_1,x_2)
	 \end{equation} 
 \end{definition}
 \begin{equation}
 p_1 x_1^{*}  + p_2 x_2^{*} = m
 \end{equation} 
 \begin{equation}
 x_2^{*} =  a x_1^{*}
 \end{equation} 
 \begin{equation}
 x_1^{*}  = \frac{m}{p_1 + a p_2} ; x_2^{*} = \frac{am}{p_1 + ap_2}
 \end{equation} 
 Набор из 1 единицы блага 1 и $a$ единиц блага 2 стоит  $p_1 + a p_2$. Оптимальными являютя наборы $\frac{m}{p_1 + a p_2}$
 \section{Индвидуальный спрос}
 Анализ сравнительной статики функций индвидуального спроса --
 исселдование характера измененния спроса потребителя
 на представленные в наборе блага
 $x_1^{*}(p_1,p_2,m)$, $x_2^{*}(p_1,p_2,m)$ на вариацию в значения рыночных цен $p_1,p_2$ дохода потребителя $m = y$
 \subsection{кривая цена потребления для Кобба Дугласа}
 \begin{equation}
 U(x_1,x_2) = x_1^{\alpha} x_2^{\beta}
 \end{equation} 
 \begin{equation}
 x_1^{*} (p_1,p_2,m) = \frac{a}{a+b} \times \frac{m}{p_1}
 \end{equation} 
 \begin{equation}
 x_2^{*} (p_1,p_2,m) = \frac{b}{a+b} \times \frac{m}{p_2}
 \end{equation} 
 Кривая цена-потребления для блага 2 по  $p_1$ есть прямая линия,
 для блага $1$ она гипербола.
 \subsection{Взаимо дополняющие блага }
 Если функция такая, то
 \begin{equation}
	 U = \max{x_1,x_2}
 \end{equation} 
 \begin{equation}
 x_1^{*}(p_1,p_2,m) = x_2^{*}(p_2,p_2,m) = \frac{m}{p_1 + p_2}
 \end{equation} 
 При заданных $p_2,m$ большие значения $p_1$ приводят к меньшим
 занчениям $x_1^{*},x_2^{*}$
 \begin{equation}
 p_1 \to 0, x_1^{*} = x_2^{*} \to \frac{m}{p_2}
 \end{equation} 
 \begin{equation}
 p_1 \to \infty , x_1^{*} = x_2^{*} \to 0
 \end{equation} 
 \subsection{Изменение спроса на благо по собственной цене}
 Рассматривая востребованное количечество блага
 в качестве заданной величины и выясняя цену, по
 которой потребитель
 \begin{equation}
 x_1^{*} = \frac{a m}{(a + b) p_1}
 \end{equation} 
 \begin{equation}
 p_1 = \frac{a m}{(a + b) x_1^{*}}
 \end{equation} 
 \begin{equation}
 x_1^{*} = \frac{y}{p_1 + p_2}  \implies \frac{y}{x_1^{*}} - p_2
 \end{equation} 
% \begin{tikzpicture}
% 	\draw[->] (0,0) -- (0,10);
% 	\draw[->](0,0) -- (10,0);
% \end{tikzpicture}
 \begin{definition}[Кривая Энгеля]
	 График зависимости 
	 между величиной спроса потребителя на благо
	 от величины его дохода называется
	 кривой Энгеля
 \end{definition}
 \subsection{Изменение по доходу в случае предпочтений Кобба-Дугласа}
 \begin{equation}
 y = \frac{( a + b )p_1}{a} x_1^{*}
 \end{equation} 
 \begin{equation}
 y = \frac{( a + b ) p_2}{b} x_2^{*}
 \end{equation} 
 Если функция полезность $U = \min {x_1,x_2}$
 \begin{equation}
 y = (p_1 + p_2) x_1^{*}
 \end{equation} 
 \begin{equation}
	 y = (p_1 + p_2) x_2 ^{*}
 \end{equation} 
 прямые тоже
 \begin{equation}
 U(x_1,x_2) = x_1 + x_2
 \end{equation} 
 \begin{equation}
 x_1^{*}(p_1,p_2,y) =
 \begin{cases}
 	0, p_1 > p_2\\
	\frac{y}{p_1} p_1 < p_2
 \end{cases}
 \end{equation} 
 \begin{equation}
 x_w^{*}(p_1,p_2,y) =
 \begin{cases}
 	0, p_1 < p_2\\
	\frac{y}{p_2} p_1 > p_2
 \end{cases}
 \end{equation} 
 \begin{equation}
 x_1^{*} = \frac{y}{p_1} ,x_2^{*} = 0 , p_1 < p_2
 \end{equation} 
 Прямая Энгеля представляется в форме
 линейной зависимости в случае, когда отношение
 предпочтение является гомотетичным
 \begin{definition}[Гомотичное отношение]
	 $\forall  k > 0$ 
	 \begin{equation}
		 (x_1,x_2) \to (y_1,y_2) \equiv
		 (kx_1,kx_2) \to (ky_1,ky_2)
	 \end{equation} 
	 Гомотетичность отношения предпочтения
	 означает,  что предельная норма замены
	 благ $\mrs$ неизменная вдоль лучей, исходящих
	 сз начала координат
 \end{definition}
 \subsection{Пример негомотичного отношения}
 \begin{equation}
 U = f(x_1) + x_2
 \end{equation} 
 \begin{definition}
	 Функция однородна степени $k$  если $f(\lambda x) = \lambda^{k} f(x)$
 \end{definition}
 \begin{definition}[Линейно однородная функция]
	 $f(\lambda x) = \lambda f(x)$
 \end{definition}
 \begin{definition}[Однородность нулевой стеени]
	 $f:\mathbb{R}^{N} \to \mathbb{R}, \forall  \lambda > 0 ,f(\lambda x) = f(x)$
 \end{definition}
 \begin{definition}[Гомотетичная функция]
 	$f: \mathbb{R}^{N} \to \mathbb{R}$ 
	называется гомотетичной, если она представима в виде
	$f(x) = g(h(x))$ ,  $g\mathbb{R} \to \mathbb{R}$ 
	является строго возрастаяющей, а $h:\mathbb{R}^{N} \to \mathbb{R}$ является линейно однородной
 \end{definition}
 Обозначим через $x^{*}_{i} (p_{i}) $ 
 функцию задающую кривую спроса, при фиксированных значениях 
 всех цен, коме $i$-ой и дохода  $y$

 \subsection{Эффект дохода}
 Если для блага ( товара )  $i$ выполняется
 $\frac{\Delta x_{i}^{*}}{\Delta y} > 0$ 
 спрос растет при увелчении потребителя, то оно называется
 нормальным, кривая Энегеля имеет положительный наклон.

 Если $\frac{\Delta x_{i}^{*}}{\Delta y}$ то он малоценный по 
 доходу. Кривая энгеля имеет отрицательный наклон

 Благо $i$ называется обычным, если с увеличением его собственной
 цены  $p_{i}$ цена  спрос убывает. Кривая спроса имеет отрицательный наклок

 Если при увелечение цены, спрос растет, то такое благо называется
 Гиффиновым благом.

 \subsection{Кривые Торнкиста}
 \begin{enumerate}
 	\item Рассматривается кривая, которая описывает
		изменение спроса на потребительские товары в зависимости от дохода потребителя
	\item Обозначм доход потребителя через $y$,
		величину предъявляемоего спроса через  $x$
	\item В соответсвии с конфигурацией кривой спроса 
		выделяются три группы товаров на рынке.
 \end{enumerate}
 \subsubsection{Товары первой необходимости}
 \begin{equation}
 x_1(y) - \frac{a_1 y}{x_1 + y}
 \end{equation} 
 Спрос
  \begin{enumerate}
 	\item $y$ доход потребителя
	\item  $x_1$ объем спроса на товары первой необходимости
	\item $a_1,c_1$ параметры зависимости $a_1>0,c_1>0$
 \end{enumerate}
 \subsubsection{Товары длительного пользования}
 \begin{equation}
 x_2(y) = 
 \frac{a_2 (y - M_2)}{c_2 + y}
 \end{equation} 
 \begin{enumerate}
	 \item $y$ доход потребителя  $y>M_2$
	\item $x_2$ объем спроса на предметы длительного 
		пользования
	\item $a_2,c_2,M_2$ параметры зависимости $a_2>a_1>0, c_2 >0, M_2 > 0$
 \end{enumerate}
 \subsubsection{Предметы Роскоши}
 \begin{equation}
 x_3(y) = \frac{a y (y - M_3)}{c_3 + y}
 \end{equation} 
 \begin{enumerate}
 	\item $y$ -- доход потребителя  $y > M_3$
	\item $x_3$ объем спроса на предметы роскоши
	\item $a_3,c_3,M_3$ параметры зависимости причем 
		$a_3>0,c_3>0,M_3>M_2>0$
 \end{enumerate}
 \section{Хикс}
 Эффект цены заключается в том, что
 потребитель товара $b$,
 сокращает его потребление.
 При этом на него воздействуют
 два эффект -- эффект дохода 
 реальный доход снижается. 
 Эффект замены (один из товаров становится более дорогим по
 отношению к другому)

 \subsection{Логика Хикса}
 Давайте дадим потребителю
 денег, чтобы он сохранил потребление.
 Потребителю нужно скопенсировать подорожание
 $D$ чтобы уровень полезности не изменился
 \subsection{Логика Слуцкого}
 По логике Слуцкого потребитель должен получить
 компенсации и потреблять тот же самый товарный набор $H$.
 Количественно эффекты по Слуцкому и Хиксу не совпадают

 Содержательно они совпадают
\end{document} 
