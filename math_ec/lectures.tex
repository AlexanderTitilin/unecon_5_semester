\documentclass[14pt]{extarticle}
\usepackage{fontspec}
\usepackage[russian, english]{babel}
\setmainfont{Times New Roman}
\usepackage{amssymb}
\usepackage{setspace}
\onehalfspacing
\usepackage{amsmath}
\usepackage{amsthm}
\usepackage{listings}
\usepackage{indentfirst}
\setlength{\parindent}{1.25cm}
\usepackage[right=10mm,left=30mm,top=20mm,bottom=20mm]{geometry}
\newtheorem{theorem}{Теорема}
\newtheorem{definition}{Определение}
\newtheorem{example}{Пример}
\newtheorem{corollary}{Следствие}[theorem]
\newtheorem{lemma}[theorem]{Лемма}
\DeclareMathOperator{\mrts}{MRTS}
\DeclareMathOperator{\mrs}{MRS}
\DeclareMathOperator{\wpp}{WP}
\title{Математическая экономика}
\author{}
\date{}
\begin{document}
	\maketitle
	\begin{enumerate}
		\item Дмитририев Антон Леонидович
		\item dmitr7171@mail.ru
		\item Элементы математической экономики, Экланд (как сказка на ночь)
		\item Аллен Р.Г Математическая экономия
		\item Тарасевич Микроэкономика
		\item Микроэкономика практикум ii (Дмитриев)
	\end{enumerate}
	\section{Предпочтения}
	Нам надо научиться математически
	моделировать индивида.
	\begin{definition}[Поведенческий Постулат]
		Лицо принимающее решение
		всегда выбирает наиболее предпочтительную
		для себя альтернативу
	\end{definition}
	Модель выбора должна содержать:
	\begin{enumerate}
		\item описание системы предпочтений ЛПР
		\item множество альтернатив, доступных ЛПР
	\end{enumerate}
	\begin{enumerate}
		\item В основе выбора
			наилучшей альтернативы лежит
			сравнение возможных вариантов
		\item сравнение любых
			альтернатив предполагает их прямое или
			косвенное сопоставление
		\item ЛПР сравнивают любую пару
			возможных вариантов по приницу
			лучше хуже. С точке срение математики 
			задается бинарное отношение.
	\end{enumerate}
		Пусть $М$
		множество непустое ЛПР
		альтернатив. Рассмотрим
		множество всех упорядоченных пар  $(x,y)$,
		 $(x,y) \sim (y,x)$ безразличие
	 \begin{definition}[Бинарное отношение]
		 \begin{equation} 
		 A \subseteq M \times M
		 \end{equation} 
	\end{definition}
	\begin{definition}[Функция]
		$\forall x \in M \exists ! y \in M$ для которого
		справедливо $x A y$
	\end{definition}
	\begin{definition}[График]
		\begin{equation} 
			\Gamma = \{(x,y) \mid x,y \in M, xAy\} 
		\end{equation} 
	\end{definition}
	\begin{definition}[Отношение предпочтения]
		Потребитель сравнивает два набора благ
		\begin{enumerate}
			\item строгое предпочтение ($x$ лучше  $y$)
			\item слабое предпочтение $x$ не хуже  $y$
			\item  безразличие,  $x$ и  $y$ одинаково хороши
		\end{enumerate}
	\end{definition}
	Рассмотрим символьную запись
	\begin{enumerate}
		\item $x \succ y$  $x$ строго лучше  $y$  
		\item $x \succeq y$ $x$ не хуже  $y$
		\item  $x \sim y$  $x,y$ одинакого предпочтительны
	\end{enumerate}
	\subsection{Гипотезы (аксиомы) о свойствах}
	\begin{enumerate}
		\item \textbf{ Полнота } для любых наборов выполняется $x \succeq y$ или $y \succeq x$
		\item \textbf{Рефлексивность} для любого $x$ ,  $x \succeq x$
		\item \textbf{Транзитивность},
			есть три набора благ
			$x,y,z$
			 \begin{equation} 
			x \succeq  y \land y \succeq z \implies x \succeq z
			\end{equation} 
	\end{enumerate}
	\subsection{Непрерывность отношения}
	\begin{definition}
		Отношение на множестве $X$ 
		непрерывно
		если для любого вектора
		 $y \in X$  
		 множества
		  \begin{equation} 
			  \{x \in X \mid x \succeq y\} 
		 \end{equation} 
		 \begin{equation} 
			 \{x \in X \mid x \preceq y\} 
		 \end{equation} 
		 являются замкнутыми
	\end{definition}
	\begin{definition}[Рациональное отнощение потребления]
		Определенное на множестве
		наборов благ $R^{n+}$ 
		отношения предпочтения $\succeq$ 
		называется
		рациональным, если оно является:
		\begin{enumerate}
			\item полным
			\item рефлексивным
			\item транзитивным
		\end{enumerate}
	\end{definition}
	\subsection{Свойства рационального отношения предпочтения}
	В случае рациональность $\succeq$ :
	\begin{enumerate}
		\item $\succ$ антирефлексвно  ( не выполняется $x \succ x$ ), транзитивно
		\item $\succeq$ рефлексивно,транзитивно,симметрично
		\item $x \succ y \succeq z \implies x \succ z$
	\end{enumerate}
	\subsection{Кривые безразличия}
	\begin{enumerate}
		\item Зафиксируем некоторый набор благ
			$x'$ .
			Множество всех набор одинаково
			предпочтительных с  $x'$,
			называется кривой безразличия, содержашей  $x$
	\item Посколько кривая не сегла кривая в геометрическом
		смысле, то правильно говорить о
		множестве безразличия
		\begin{equation} 
			I(x') = \{y \in R^{n+} \mid y \sim x'\} 
		\end{equation} 
	\end{enumerate}
	\begin{equation} 
	\wpp(x) 
	\end{equation} 
	множество наборов не хуже $x$,  $I(x) \subseteq \wpp(X)$

	Кривые безразличя не пересекаются, возникнет нарушение транситивности
	\subsection{Наклон кривых безразличия}
	\begin{enumerate}
		\item Товар наличие которого в большем количестве
			всегда предпочтительнее меньшего, называется
			\textbf{благом}
		\item Если в наборе присутствуют только блага,
			то кривая безразличия имеет отрицательный
			наклон, по отношению к 
			осям соответсвующих благ
	\end{enumerate}
	\begin{definition}[Антиблаго]
		Товар, наличие которого
		в наборе в меньшем количестве всегда предпочтительнее большего называется \textbf{антиблаго}
	\end{definition}
	\begin{definition}[Совершенные заменители]
		Если потребитель в любых
		условиях считает два блага эквивалетными,
		то он совершенные заменители.

		Если набор состояит из совершенных
		заменителей, то предпочтительность определяется
		общим количеством.
	\end{definition}
	\begin{definition}
		Если потрибитель во всех ситуациях
		исползует блага 1,2 в некоторой
		фиксированной пропорции,
		такие блага называются \textbf{совершенными дополняемыми}
		\begin{equation} 
			U = \min(x_1,x_2)
		\end{equation} 
	\end{definition}
	\begin{definition}
		Набор благ, строго предпочитаемый всем
		другим, называется точкой насыщения
	\end{definition}
	\begin{definition}
		Отношение предпочтения
		мы будем называть локально ненасышенным
		дл любого набора
		$x \subseteq R^{n+}$ 
		и произвольного числа $t > 0$ 
		найдется  $y \subseteq R^{n + }$ 
		$y - x \preceq t$ и при этом $y \succeq x $
	\end{definition}
	\begin{enumerate}
		\item Бесконечно делимое благо
		\item Дискретное благо
	\end{enumerate}
	\begin{definition}
		Рациональное отношение 
		предпочтения, является регулярным
		если оно
		\begin{enumerate}
			\item монотонно, большее 
				количество блага всегда предпочитается меньшему (наборы состоят только из благ)
			\item выпуклое
		\end{enumerate}
	\end{definition}
	Выпуклая комбинация двух различных, но при этом
	одинаково предпочтительных наборов предпочтительных или по крайней мере не хуже, чем каждый из сотавляющих наборов
	\subsection{Наклон кривых безразличия}
	Вычисленный в конкретной тчоке наклон,
	кривой безразличия характеризует в ней 
	предельную норму замены благ MRS (margina rate
	of substitution)

	$\mrs$ в точке  $x'$ 
	харакетризует исчисленный в ней наклон
	кривойй безразличия, которому эта точка принадлежитю
	Геометрически  $\mrs$ есть тангенс угла наклона касательной
	и кривой без различия в точке  $x'$

	$\mrs$ в точке  $x'$ есть  $\lim_{\Delta x_1 \to 0} \frac{\Delta x_2}{\Delta x_1}$ или иначе $\frac{d x_2}{d x_1}$ в точке $x'$


	Если набор составлен из двух благ, то соответсвующие кривые
	имеют отрицательный наклон  $\mrs < 0$

	Если набор включает одно благо и одно антиблаго $\mrs > 0$
	\section{Полезность}
	\begin{definition}
		Функция полезност $U(X)$ описывает отношение
		предпочтения  $\succeq$ тогда
		и только когда для двух наборов благ  $x', x''$ 
		верны следущие соотношения
		 \begin{enumerate}
			\item $x' \succ x'' \iff U(X') > U(x'')$
			\item  $x' \prec x'' \iff U(x') < U(x'')$
			\item  $x' \sim x'' \iff U(x') = U(x'')$ 
		\end{enumerate}
	\end{definition}
	Непрерывное, монотонное возрастающее рциональное
	отношение педпочтения может
	быть представлено непреывной функцией полезности

	\begin{definition}[непрерывность]
		Малые изменения набора благ, ведут малые изменения
		предпочтительности набора благ
	\end{definition}
	Полезность является порядковым (задающим упорядочение) 
	понятием
	\begin{example}
		$U(x) = 6,U(y)=2$ x строго предпочтительнее $y$,
		но при этом нельзя сказать, что $x$ в 3 раза
		предпочтительнее  $y$
	\end{example}
	\begin{example}
		Рассмотрим наборы благ, представленные векторами
		$(4,1),(2,3),(2,2)$
		
		Допустим 
		 \begin{equation}
			 (2,3) \succ (4,1) \sim (2,2)
		\end{equation} 
		Поставим в
		соотсветвие этим  наборам произвольные числа,
		сохранябщие упорядчение векторов по предпочтительности
		\begin{equation}
		U(2,3) = 6 > U(4,1) = U(2,2) = 4
		\end{equation} 
		Назовем эти числа уровнями полезности
	\end{example}
	\begin{definition}
		Совокупность всех кривых безразличия
		называется картой кривых безразличия
	\end{definition}
	\begin{example}
		Пусть $U(x_1,x_2) = x_1 x_2$ описывает
		отношение $\succeq$
		\begin{equation}
		V = U^2 = x_1^2 x_2 ^2
		\end{equation} 
		\begin{equation}
			V(2,3) = 35 > V(4,1) = V(2,2) = 16
		\end{equation}
		\begin{equation}
			(2,3) \succ (4,1) \sim (2,2)
		\end{equation} 
		$V$ сохраняет тоже самое упорядочение, что и
		 $U$. Функция описывает одинаковое с  $U$ отношение предпочтения
		  \begin{equation}
		 W = 2 U +10
		 \end{equation} 
		 \begin{equation}
			 W(2,3) = 22 > W(4,1) = W(2,2) = 18
		 \end{equation} 
	\end{example}
	Если $U(x)$ является функцией полезности
	описыващей отношение предпочтения  $\succeq$
	на множестве неотриательных наборов блаш  $R^{n+}$ ,
	$f(U)$ есть строго возрастающая функция, одного
	аргумента, то зависимость  $V = f(U)$ так же
	представляет собой функцию полезности,
	описывающую исходное отношение предпочтения  $\succeq$

	\begin{theorem}[о существовании непрерывной функции полезности]
	Пусть отношения предпочтения ЛПР $\succ$
	является полным, рефлексивным, непрерывным и строго монотонным.
	Тогда существует непрерывная функция полезности  $U:R^{n+} \to R$ описывающая данное отнощение предпочтения
	\end{theorem}
	Рассмотрим $V = x_1 + x_2$ ее кривые безразличия это
	прямые линии, состоящих из совершенных заменителей

	Рассмотрим $W(x_1,x_2) = \min(x_1,x_2$ ее кривые безразличия
 блага совершенные дополнители


Рассмотрим $U(x_1,x_2) = f(x_1) + x_2$ квазилинейная функция, является линейной только по $x_1$
\begin{definition}[Функция Кобба-Дугласа]
	\begin{equation}
		U(x_1,x_2) = x_1^{a} x_2^{b}, a>0,b>0
	\end{equation}
\end{definition}
\begin{definition}[Предельная Полезность]
	Предельная полезность продукта $i$
	\begin{equation}
		MU_{i} = \frac{\partial U}{\partial x_{i}}
	\end{equation} 
\end{definition}
\begin{example}
	\begin{equation}
	U(x_1,x_2) = x_1^{\frac{1}{2}}x^2_2
	\end{equation} 
	\begin{equation}
	MU_1 = \frac{1}{2} x_1 ^{- \frac{1}{2}}x^2_{2}
	\end{equation} 
	\begin{equation}
	MU_1 = 2 x_1^{\frac{1}{2}} x_2
	\end{equation} 
\end{example}
Общее уравнение кривой безразличия функции 
полезности $U(x_1,x_2)$ иметт вид $U(x_1,x_2) = k, k > 0 , k = \text{const}$. Полный дифференциал
\begin{equation}
\frac{\partial U}{\partial x_{1}} dx_1 + \frac{\partial U}{\partial x_2} dx_2 = 0
\end{equation} 
\end{document}
