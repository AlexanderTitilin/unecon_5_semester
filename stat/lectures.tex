\documentclass[14pt]{extarticle}
\usepackage{fontspec}
\usepackage[russian, english]{babel}
\setmainfont{Times New Roman}
\usepackage{amssymb}
\usepackage{setspace}
\onehalfspacing
\usepackage{amsmath}
\usepackage{listings}
\usepackage{indentfirst}
\setlength{\parindent}{1.25cm}
\usepackage[right=10mm,left=30mm,top=20mm,bottom=20mm]{geometry}
\newtheorem{theorem}{Теорема}
\newtheorem{corollary}{Следствие}[theorem]
\newtheorem{lemma}[theorem]{Лемма}
\title{Матстата}
\author{}
\date{}
\begin{document}
	\maketitle
	\section{}
	Открыть старый канспект, повторить распределения, несобственные интегралы.
	\section{Отступление матан. Гамма-функция Эйлера}
	\begin{eqnarray}
		\Gamma(y) = \int\limits_{0}^{+\infty}  t^{y-1}e^{-t}dt,\\ f(t) = t^{y-1}e^{-t}
	\end{eqnarray}
	Рассмотрим $t^{y+1} e^{-t} = \frac{t^{y+1}}{et} \to 0,$ при $t \to + \infty$  $\implies$  $\exists  A: \forall t \ge  A ~ t^{y+1} e^{-t} < 1 \implies f(t) = t^{y-1} e^{-t} < \frac{1}{t^2}$
	\[
	\int\limits_{0}^{+\infty}  f(t) dt = \int\limits_{0}^{A} f(t)  dt + \int\limits_{A}^{+\infty}  f(t) dt = I_1 + I_2
	.\] 
	$I_2$ сходится при $\forall  y \in \mathbb{R}$, по первому признаку сравнения с $\int\limits_{A}^{+\infty} g(t) dt , g(t) = \frac{1}{tst}  $
	
	Рассмотрим $I_1$  при $y < 1$ это несобственный интеграл
	 \[
		 t^{y-1}e^{-t}< t^{y-1}
	.\] 
	$\int\limits_{0}^{A} \frac{dt}{t^{1- y}}  $ сходится при $y > 0$ ,  $1- y < 1$
	
	\textbf{Вывод}. При $y>0$  $\Gamma(y) =  \int\limits_{0}^{+\infty} t^{y-1}e^{-1} dt $
	
	Без доказательства для $n \in \mathbb{N}$
	\begin{equation}
		\Gamma(n + \frac{1}{2}) = 2^{-n} \sqrt{\pi} (2n -1)!!
	\end{equation}
	\begin{equation}
		\Gamma(y+1) \int\limits_{0}^{+\infty}  t^{y}e^{-t}dt
		- \int\limits_{0}^{+\infty}  t^{y} d e^{-t}=
		-t^{y}e^{-t} \mid_{0}^{+\infty} + y \int\limits_{0}^{+\infty} y^{-1} e^{-t} dt = 0 + y \Gamma(y)
	\end{equation}
	\begin{equation}
		\Gamma(y+1) = y \Gamma(y)
	\end{equation}
\begin{eqnarray}
	\Gamma(1) = \int\limits_{0}^{+\infty}  e^{-t} dt = -e^{-t}\mid_{0}^{+\infty}\\
	\Gamma(2) = 1*\Gamma(1) =1\\
	\Gamma(3) = 2*\Gamma(2) = 2\\
	\Gamma(n+1) = n!
\end{eqnarray}
Можно показать, что Гамма функция дифференцируема
\begin{equation}
		\Gamma(\frac{1}{2}) = -\frac{1}{2}\Gamma(-\frac{1}{2})
		\implies \Gamma(-\frac{1}{2}) = -2 \Gamma(\frac{1}{2})
\end{equation}
Но 
\begin{equation}
	\Gamma(0) = + \infty , ~ \int\limits_{0}^{A}   t^{-1}e^{-t} = \int\limits_{0}^{A}  \frac{dt}{t^{et}}
\end{equation}
По второму признаку сравнения это расходится
\[
\lim_{t \to 0+0}  \frac{1}{te^{t}} : \frac{1}{t} = \lim_{t \to 0+0} 1 \in (0;+\infty)
.\] 
\textbf{Вывод.} $\Gamma$ функцию можно продолжить на  $\mathbb{R}^{-}$ ккроме целых точек
\begin{equation}
	C_{n}^{k} = \frac{n!}{k! (n-k)!} = \frac{\Gamma(n+1)}{\Gamma(k+1)\Gamma(n - k + 1)}
\end{equation}
Мы расшили понтие числа сочений
 \begin{equation}
 	C_{\alpha}^{\beta} = 
	\frac{\Gamma(\alpha + 1)}{\Gamma(\beta + 1) \Gamma(\alpha -\beta + 1)} ,  \alpha,\beta \in \mathbb{R} \setminus \mathbb{Z}^{-}
 \end{equation}
 \section{Закон распределения Лапласа (двойное экспонициальное распределение)}
 Он применяетс для моделирования обработки сигналов, в моделировании биологических процессов, экономике и финансах

 Это распределние НСВ $X$ с плотностью
  \begin{equation}
 	f(x) = \frac{\lambda}{2} e^{-\lambda |x|}, x \in \mathbb{R},\lambda > 0
 \end{equation}
 \begin{equation}
 	F(x) =
	\begin{cases}
		\frac{1}{2}e^{\lambda x} , x \le 0\\
		1 - \frac{1}{2}e^{-\lambda x}
	\end{cases}
 \end{equation}
 \begin{equation}
	 x\le  0,
 	F(x) = \int\limits_{-\infty}^{x}   \frac{\lambda}{2} e^{\lambda t} dt = \frac{1}{2} e^{\lambda t} \mid_{-\infty}^{x} = \frac{1}{2}(e^{\lambda x} - 0)
 \end{equation}
 \begin{equation}
 	x > 0 =  F(0)  + \int\limits_{0}^{x}
 \end{equation}
 из вида $f(x)$ :  $Mo(X) = Me(X) = M(X) = 0$
 \begin{eqnarray}
 	D(X) = M(X^2) = 
	\int\limits_{-\infty}^{+\infty}  x^2 f(x) dx = 
	2 \int\limits_{-\infty}^{\infty} x^2 * \frac{\lambda}{2} e^{-\lambda x} dx = \dots = (- x^2 +\frac{2}{\lambda} x
	+ \frac{2}{\lambda^2}) e^{-\lambda x} \mid_{0}^{+\infty}\\
 \end{eqnarray}
 \begin{equation}
  = \frac{2}{\lambda^2}
 \end{equation}
 \section{Распредение Вейбула}
 Это распредение имеют времена безотказной работы технческих устройтв.
 В таких значениях важной характеристикой является интенсивность отказа $k(t)$ 
  \begin{equation}
	  k(t) = - \frac{ [P(X \ge  t)]'}{P(X \ge  t)}, k(t) = \frac{f(t)}{1 - F(t)}
 \end{equation}
 Получили диффур. Это УРП, решаем элементарно
 \begin{eqnarray}
 	k(t) = \frac{y'}{1 - y}\\
	k(t) dt = \frac{dt}{1 -y}\\
	-\int k(t)dt + C = \ln{( 1-y )}\\
	y = 1 - e^{-\int k(t) dt + C}\\
	y(0) = 0 \implies y = 1 -e^{-\int\limits_{0}^{x} k(t)dt }
 \end{eqnarray}
 Во многих случаях график $k(t)$ имеет следущий вид
  \begin{enumerate}
 	\item период обкатки
	\item период нормальной эксплуатации
	\item период старения
 \end{enumerate}
 Рассмотрим класс степенных зависимостей $k(t)=\lambda \alpha t^{`-1`}$ где $\lambda > 0 , \alpha > 0$ некоторые числовые параметры.
 Периодам 1,2,3 отвечают  $\alpha < 1,\alpha =1, \alpha>1$ соответственно

 Функция распеделения
 \begin{equation}
 	F_{X}(x) = 1 - e^{- \frac{0}{x} \lambda \alpha t^{\alpha - 1} dt} = 1- e^{\lambda t^{\alpha}} \mid_{0}^{x} = 1 - e^{-\lambda x^{\alpha}}
 \end{equation}
 Плотность
 \begin{equation}
 	f_{X}(x) (F_{X}(x))' = \lambda \alpha x^{\alpha - 1} e^{-\lambda x^{\alpha}}
 \end{equation}
 При $a = 1$ получим  $E(\lambda)$ , при  $\alpha = 2$ получим распределениие Рэлея  $f(x) = 2\lambda x e^{-\lambda x^2}$
 \subsection{Числовые характеристики Распределения Вейбулла}
 \begin{equation}
 	M(X) = \lambda^{-\frac{1}{\alpha}} \Gamma(1+  \frac{1}{\alpha})
 \end{equation}
 \begin{equation}
 	M(X) = \int\limits_{0}^{+\infty}  x \alpha * \lambda x^{\alpha - 1} e^{-\lambda x^{\alpha}} dx = \int\limits_{0}^{+\infty} \lambda^{-\frac{1}{\alpha}} t^{\frac{1}{\alpha}} e^{-t} dt = \lambda^{-\frac{1}{\alpha}} \Gamma(1 + \frac{1}{\alpha})
 \end{equation}
 \begin{equation}
	 M(X^2) = \lambda^{-\frac{-2}{\alpha}} \Gamma(1 + \frac{2}{\alpha}) \implies D(X) = \lambda^{-\frac{2}{\alpha}}  (
	 	\Gamma(1 + \frac{2}{\alpha}) - \Gamma^2(1 + \frac{1}{\alpha})
	 )
 \end{equation}
 \begin{equation}
	 \text{Me} (X) = (\frac{1}{\lambda} \ln{2}) ^{\frac{1}{\alpha}}
 \end{equation}
 \begin{equation}
 	Mo(X) = 
	\begin{cases}
		0, \alpha \le  1\\
		(\frac{\alpha - 1}{\lambda \alpha})^{\frac{1}{\alpha}}, \alpha > 1
	\end{cases}
	f'(x )  = 0
 \end{equation}
 \section{Гамма-распеделение}
 Оно используется для описание времен безотказной работы различных 
 технических устройств

 Его имеет НСВ $X = \gamma(a,b)$ с плотностью
  \begin{equation}
 	f(x) =
	\begin{cases}
		\frac{b^{a}}{\Gamma(a)} x^{a - 1} e^{-b x} ,x>0\\
		0 , x \le  0
	\end{cases}
 \end{equation}
 где $a > 0$ параметр формы,  $b > 0$ параметр масштаба
 \subsection{Свойства}
  \begin{enumerate}
 	\item $b * \gamma(a,b) = \gamma(a,1)$ 
	\[
	f_{\alpha X + \beta}(x) = f_{X}( \frac{x -  b}{a} )*\frac{1}{|a|} \implies
	.\] 
	\[
		f_{b \gamma(a,b)}(x) = f_{\gamma_{a,b}}(\frac{x}{b})\frac{1}{b} = \frac{b^{a}}{\Gamma(a)} \frac{x^{a -1}}{b^{a-1} }*e^{-b \frac{x}{b}} * \frac{1}{b} = \frac{1}{\Gamma(a)} x^{\alpha-1}e^{-x} = f_{\gamma(a,1)}(x)
	.\] 
\item Если случайные величины независимы
\item $a = \frac{m}{2},b = \frac{1}{2}$ получм $\chi^{2}_{m}$,$a =1$ получим экспонициальное
 \end{enumerate}
 \subsection{Функция распределение}
 \begin{equation}
 	F_{\gamma(a,b)}) (x) = \frac{1}{\Gamma(a)} \int\limits_{0}^{bx}   \tau^{a - 1} e^{-\tau}
 \end{equation}
 \subsection{Числовые Характеристики}
 \begin{equation}
 	M(\gamma(a,b)) = \frac{b^{a}}{\Gamma(a)} \int\limits_{0}^{+\infty} x * x^{a-1}  e^{-bx} dx =_{t=bx} \frac{b^{a}}{\Gamma(a)}\int\limits_{0}^{+\infty}   \frac{t^{a}}{b^{a}} e^{-t} * \frac{dt}{b} = \frac{\Gamma(a+1)}{b\Gamma(a)} = \frac{a \Gamma(a)}{b \Gamma(a)}= \frac{a}{b}
 \end{equation}
 \begin{equation}
 	D(\gamma(a,b)) = \frac{a(a+1)}{b^{2}} - (\frac{a}{b})^{2}=
	\frac{a}{b^2}
 \end{equation}
 \begin{equation}
 	Mo = \frac{a - 1}{b} ,a\ge 1
 \end{equation}
 \begin{equation}
 	f'(x) = 0
 \end{equation}
\end{document}
