\documentclass[14pt]{extarticle}
\usepackage{fontspec}
\usepackage[russian, english]{babel}
\setmainfont{Times New Roman}
\usepackage{amssymb}
\usepackage{setspace}
\onehalfspacing
\usepackage{amsmath}
\usepackage{listings}
\usepackage{indentfirst}
\setlength{\parindent}{1.25cm}
\usepackage[right=10mm,left=30mm,top=20mm,bottom=20mm]{geometry}
\newtheorem{theorem}{Теорема}
\newtheorem{definition}{Определение}
\newtheorem{corollary}{Следствие}[theorem]
\newtheorem{lemma}[theorem]{Лемма}
\title{Математическое моделирование}
\author{}
\date{}
\begin{document}
	\maketitle
	\section{Основные понятия}
	\begin{definition}
		Модель -- образ или прообраз какого либо
		объекта или системы объектов, используется
		в качестве аналога реальной системы.
	\end{definition}
	\begin{definition}
		Математическая модель -- описание
		объекта на языке математики.
	\end{definition}
	\begin{definition}
		Математическое моделирование--
		средство исследования сложных систем и объектов
		различной природы на основе математической моделей.

	\end{definition}
	\section{Требования}
	\begin{enumerate}
		\item адекватность
		\item конечность
		\item полнота (информативность)
		\item упрощенность
		\item гибкость
		\item приемлемая трудоемокость разработки
	\end{enumerate}
	\section{Этапы построения модели}
	\begin{enumerate}
		\item Определение цели процесса моделирования
		\item Изучение предметной области, выявить причинно-следственные связи, построить концептуальную модель
		\item переход к формальному описанию
		\item проверка адекватности
		\item корректировка модели
		\item применение модели. Проведение исследований и практическое использование.
		\item уточнение  улучшение модели
	\end{enumerate}
	\section{Классификация}
	\begin{enumerate}
		\item Статические, Динамические модели;
		\item Линейные, нелинейные;
		\item Детерминированные, стотахстические;
	\end{enumerate}
	\section{Подходы к моделированию}
	\begin{enumerate}
		\item Аналитический
		\item Аналитико-эксперементальный
		\item Экспериментальный
	\end{enumerate}
	\section{Статические модели}
	Мы связывает входы системы (независимые переменные) c
	выходами. Хотим построить функцию, которая их связывает.
	\subsection{Статические модели макроэкономичесих систем}
	\subsubsection{Модель Леонтьева}
	\begin{enumerate}
		\item В экономике $n$ отраслений
		\item каждая отрасль производит 1 вид продукции потребляет другие подукты
		\item разные отрасли производят разные виды продукции
	\end{enumerate}
	\begin{enumerate}
		\item $x_{ij}$ объем продукции, произведенный
			в отрасли $i$ и потребляемой отраслью  $j$
		\item  $X_{i}$ валовый продукт отрасли $i$
		\item  $Y_{i}$ конечный продукт отрасли $i$
	\end{enumerate}
	\begin{eqnarray}
		X_{i} = \sum_{j = 1}^{n} x_{ij} + Y_{i},~ i = 1 \dots n
	\end{eqnarray}
	\begin{equation}
		a_{ij} = \frac{x_{ij}}{X_{j}} \implies X_{i}= \sum_{j=1}^{n} a_{ij} X_{j} + Y_{i}
	\end{equation}
	\begin{equation}
		X = A\cdot X + Y
	\end{equation}
	\begin{equation}
		X = (E-A)^{-1} \cdot Y
	\end{equation}
	\begin{definition}[Продуктивность]
		Матрица $A>0$ называется \textbf{продуктивной}
		если для любого вектора  $Y > 0$ 
		сущестыует решение  $X>0$ уравнения  $(E-A)X  = Y$ 
		В этом случае и модель Леонтьева называется продуктивной
	\end{definition}
	\begin{enumerate}
		\item Если хотя бы для одного
			положительного вектора $Y$ 
			уравнение  $X=AX + Y$ имеет неотрицательное
			решение  $X$, том атрица  $A$ продуктивна
		\item
			Для
			продуктивности матрицы необходимо и достаточно существование и неотрицательности матрицы  $(E -A)^{-1}$
		 \item Неотрицательная квадратная матрица $A$ 
			 продуктивна тогда и только тогда 
			 когда максимальное по модулю собственное число  $<1$ 
		\item Неотрицательая матрица $A$ 
			продуктивна тогда и только тогла
			когда матрица 
	\end{enumerate}
	\begin{definition}
		Обратная матрица Леонтьева $B = (E - A)^{-1}$--
		матрица полных затрат. Элементы этой матрицы 
		$b_{ij}$ -- количество продукции отрасли $i$ 
		используемое для производства единицы конечного продукта отрасли  $j$
	\end{definition}
\end{document}
