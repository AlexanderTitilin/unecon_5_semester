\documentclass[14pt]{extarticle}
\usepackage{fontspec}
\usepackage[russian, english]{babel}
\setmainfont{Times New Roman}
\usepackage{amssymb}
\usepackage{setspace}
\onehalfspacing
\usepackage{amsmath}
\usepackage{listings}
\usepackage{indentfirst}
\setlength{\parindent}{1.25cm}
\usepackage[right=10mm,left=30mm,top=20mm,bottom=20mm]{geometry}
\newtheorem{theorem}{Теорема}
\newtheorem{definition}{Определение}
\newtheorem{corollary}{Следствие}[theorem]
\newtheorem{lemma}[theorem]{Лемма}
\title{Математическое моделирование}
\author{}
\date{}
\begin{document}
	\maketitle
	\section{Основные понятия}
	\begin{definition}
		Модель -- образ или прообраз какого либо
		объекта или системы объектов, используется
		в качестве аналога реальной системы.
	\end{definition}
	\begin{definition}
		Математическая модель -- описание
		объекта на языке математики.
	\end{definition}
	\begin{definition}
		Математическое моделирование--
		средство исследования сложных систем и объектов
		различной природы на основе математической моделей.

	\end{definition}
	\section{Требования}
	\begin{enumerate}
		\item адекватность
		\item конечность
		\item полнота (информативность)
		\item упрощенность
		\item гибкость
		\item приемлемая трудоемокость разработки
	\end{enumerate}
	\section{Этапы построения модели}
	\begin{enumerate}
		\item Определение цели процесса моделирования
		\item Изучение предметной области, выявить причинно-следственные связи, построить концептуальную модель
		\item переход к формальному описанию
		\item проверка адекватности
		\item корректировка модели
		\item применение модели. Проведение исследований и практическое использование.
		\item уточнение  улучшение модели
	\end{enumerate}
	\section{Классификация}
	\begin{enumerate}
		\item Статические, Динамические модели;
		\item Линейные, нелинейные;
		\item Детерминированные, стотахстические;
	\end{enumerate}
	\section{Подходы к моделированию}
	\begin{enumerate}
		\item Аналитический
		\item Аналитико-эксперементальный
		\item Экспериментальный
	\end{enumerate}
	\section{Статические модели}
	Мы связывает входы системы (независимые переменные) c
	выходами. Хотим построить функцию, которая их связывает.
	\subsection{Статические модели макроэкономичесих систем}
	\subsubsection{Модель Леонтьева}
	\begin{enumerate}
		\item В экономике $n$ отраслений
		\item каждая отрасль производит 1 вид продукции потребляет другие подукты
		\item разные отрасли производят разные виды продукции
	\end{enumerate}
	\begin{enumerate}
		\item $x_{ij}$ объем продукции, произведенный
			в отрасли $i$ и потребляемой отраслью  $j$
		\item  $X_{i}$ валовый продукт отрасли $i$
		\item  $Y_{i}$ конечный продукт отрасли $i$
	\end{enumerate}
	\begin{eqnarray}
		X_{i} = \sum_{j = 1}^{n} x_{ij} + Y_{i},~ i = 1 \dots n
	\end{eqnarray}
	\begin{equation}
		a_{ij} = \frac{x_{ij}}{X_{j}} \implies X_{i}= \sum_{j=1}^{n} a_{ij} X_{j} + Y_{i}
	\end{equation}
	\begin{equation}
		X = A\cdot X + Y
	\end{equation}
	\begin{equation}
		X = (E-A)^{-1} \cdot Y
	\end{equation}
	\begin{definition}[Продуктивность]
		Матрица $A>0$ называется \textbf{продуктивной}
		если для любого вектора  $Y > 0$ 
		сущестыует решение  $X>0$ уравнения  $(E-A)X  = Y$ 
		В этом случае и модель Леонтьева называется продуктивной
	\end{definition}
	\begin{enumerate}
		\item Если хотя бы для одного
			положительного вектора $Y$ 
			уравнение  $X=AX + Y$ имеет неотрицательное
			решение  $X$, том атрица  $A$ продуктивна
		\item
			Для
			продуктивности матрицы необходимо и достаточно существование и неотрицательности матрицы  $(E -A)^{-1}$
		 \item Неотрицательная квадратная матрица $A$ 
			 продуктивна тогда и только тогда 
			 когда максимальное по модулю собственное число  $<1$ 
		\item Неотрицательая матрица $A$ 
			продуктивна тогда и только тогла
			когда матрица 
	\end{enumerate}
	\begin{definition}
		Обратная матрица Леонтьева $B = (E - A)^{-1}$--
		матрица полных затрат. Элементы этой матрицы 
		$b_{ij}$ -- количество продукции отрасли $i$ 
		используемое для производства единицы конечного продукта отрасли  $j$
	\end{definition}
\subsection{Производственная Функция}
\begin{definition}
	Производственная функция $F$ -- 
	функция выражающая зависимость между
	затратами ресурсов и объемами выпуска
\end{definition}
\begin{enumerate}
	\item $\overline{X}$  векор используемых ресурсов
	\item $\overline{Y}$ объемы выпуска продукции каждого вида
\end{enumerate}
\subsection{Свойства неоклассических производственных функциий}
$F(x_1,x_2,\dots,x_{n})$ должна быть достаточно гладкой
хотя бы дважды дифферинцируема
\begin{definition}[Производственная функция]
	\label{pf}
	\begin{enumerate}
		\item $F(X) = 0$ ,ecли  $\exists  x_{i} = 0$
		\item $F$ возрастает по каждому аргументу  $\frac{\partial F}{\partial x_{i}}> 0$ 
		\item Выпуск по каждому аргументу неограничен
		\item $\frac{\partial ^2 F}{\partial x_{i}^2} < 0$
	\end{enumerate}
\end{definition}
\begin{definition}[Однородная функция]
	\begin{equation}
	F(\lambda x_1,\dots,\lambda x_{n}) = \lambda^{\gamma}(x_1,x_2,\dots,x_{n})
	\end{equation} 
\end{definition}
\begin{definition}[Линейная однородность]
	\begin{equation}
	F(\lambda X) = \lambda F(X)
	\end{equation} 
\end{definition}
\begin{enumerate}
	\item Целевой показатель $Y$ валовый внутренний продукт
\end{enumerate}\
\begin{enumerate}
	\item $K$ основные проиводственные фонды
	\item  $L$ число занятых
\end{enumerate}
$F(K,L)$ соответвует \ref{pf}
\begin{equation}
Y = A* K^{\alpha} L^{\beta}
\end{equation} 
\begin{enumerate}
	\item $A > 0$
	\item  $0 < \alpha < 1$
	\item  $0 < \beta < 1$
\end{enumerate}
Это производственная функция Кобба-Дугласа
\begin{equation}
Y = A K^{\alpha} L^{\beta}
\end{equation} 
\begin{enumerate}
	\item Для оценки будем использовать метод наименьших квадратов;
	\item Потребуются исторические данные $(K_{i},L_{i},Y_{i})_{i}^{i = M}$ 
	\begin{equation}
		\ln{Y} + \alpha + \alpha\ln{K} + \beta \ln{L}
	\end{equation} 
	Мы выполнили линерализацию
\end{enumerate}
\begin{equation}
S(A,\alpha,\beta) = 
\sum_{i = 1}^{M} \left(\ln \alpha + \alpha \ln K_{i} + \beta \ln L_{i} -
\ln Y_{i}\right)^2  \to \min
\end{equation} 
\begin{equation}
	\begin{cases}
	\frac{\partial S}{\partial \ln A} = 0\\
	\frac{\partial S}{\partial \alpha}  = 0\\
	\frac{\partial S}{\partial \beta}  = 0
	\end{cases}
\end{equation} 
\begin{equation}
\begin{cases}
	2\sum_{i = 1}^{M} (\ln{A} + \alpha \ln{K_{i}} +
	\beta \ln L_{i}  - \ln Y_{i})*1 = 0\\
	2 \sum_{i = 1}^{M} (\ln A + \alpha \ln K_{i} + \beta
	\ln L_{i} -  \ln Y_{i}) * \ln K_{i} = 0\\
	2 \sum_{i = 1}^{M} (\ln A + \alpha \ln K_{i} + \beta
	\ln L_{i} -  \ln Y_{i}) * \ln L_{i} = 0\\
\end{cases}
\end{equation} 
\begin{equation}
\begin{cases}
	M \ln A + \alpha \sum_{i =1}^{M} \ln K_{i}A + 
	\beta \sum_{i=1}^{M} \ln L_{i} = \sum_{i=1}^{M} \ln{Y_{i}}\\
	\ln A \sum_{i=1}^{M} \ln K_{i} +
	\alpha \sum_{i =1}^{M} \ln^2 K_{i} +
	\beta \sum_{i = 1}^{M} \ln K_{i} \ln L_{i} = \sum_{i=1}^{M} \ln Y_{i} \ln K_{i}\\
	\ln A \sum_{i=1}^{M} \ln L_{i} + 
	\alpha \sum_{i=1}^{M}\ln K_{i} \ln L_{i} +
	\beta \sum_{i=1}^{M} \ln^2 L_{i} =
	\sum_{i =1}^{M} \ln{Y_{i}}\ln L_{i}\\
\end{cases}
\end{equation} 
Рассмотрим такую 
\begin{equation}
Y = A K^{\alpha} * L^{1 - \alpha}
\end{equation} 
\begin{equation}
	\ln Y = \ln{A} + \alpha \ln{K} +(1-\alpha) \ln{L} =
	\ln A + \alpha (\ln \frac{K}{L}) + \ln{L}
\end{equation} 
\begin{equation}
S(A,\alpha) = 
\sum_{i = 1}^{M} (\ln A + \alpha \left(\frac{\ln K_{i}}{\ln L_{i}}\right)- \ln Y_{i} + \ln L)^2 \to \min
\end{equation} 
\subsection{Изокванты}
множество наборов при которых уровень производства не меняется.

$K-L$ изокватны -- линии равного уровня выпуска продукции

\begin{equation}
c_1 = A K^{\alpha} * L^{\beta}
\end{equation} 
\begin{equation}
K^{\alpha} = \frac{c_1}{A L^{\beta}}
\end{equation} 
\begin{equation}
K = \left(\frac{c_1}{A}\right)^{\frac{1}{\alpha}} * L^{\frac{- \beta}{\alpha}}
\end{equation} 
\section{Практика}
две модели $\alpha + \beta = 1, \alpha +\beta \neq b$
\begin{enumerate}
	\item Реализоать поиск $A,\alpha,\beta$ 
	\item  Вывести таблцу
	\item график зависимоти $Y$ от  $K$, исторические данные
	\item построить 3 изокванты для  $\alpha + \beta \neq 1$
\end{enumerate}
\end{document}
