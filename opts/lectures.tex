\documentclass[14pt]{extarticle}
\usepackage{fontspec}
\usepackage[russian, english]{babel}
\setmainfont{Times New Roman}
\usepackage{amssymb}
\usepackage{setspace}
\onehalfspacing
\usepackage{amsmath}
\usepackage{listings}
\usepackage{indentfirst}
\setlength{\parindent}{1.25cm}
\usepackage[right=10mm,left=30mm,top=20mm,bottom=20mm]{geometry}
\newtheorem{theorem}{Теорема}
\newtheorem{definition}{Определение}
\newtheorem{example}{Пример}
\newtheorem{corollary}{Следствие}[theorem]
\newtheorem{lemma}[theorem]{Лемма}
\DeclareMathOperator{\extr}{extr}
\DeclareMathOperator{\grad}{\textbf{grad}}
\DeclareMathOperator{\sgn}{sgn}
\DeclareMathOperator{\rank}{Rank}
\newcommand{\pa}[2]{ \frac{\partial #1}{\partial #2}}
\newcommand{\bm}[1]{ \left.#1\right|}
\title{Опты}
\author{}
\date{}
\begin{document}
	\maketitle
	\section{}
	Что мы не делали, все что не делается, предполагается
	что это делается наилучшим образом. Естественное поведение.
	Что значит наилучшим образом, мы должны установить цель.
	Эта цель должна быть достигнута наиболее эффективно.
	
	Обычно в экономике это прибыль, быстрая реализация прибыли,
	минимализация вреда окружающей среде. Мы определяем критерий качества.
	Критериев может быть несколько. Мы должны выбрать переменные,
	понять их смысл, чтобы достигнуть максимально эффективно цели.

	Окружающий мир ставит рамки (ограничены
	в финансах, энергии, площадь производства)

Gg	Возникает следущая задача -- реализовать проект
	наиболее качественно, не нарушая ограничений.

	Нужно составить математическую модель. Не для всякой
	можно построить математическую модель. Мат модель надо строить аккуратно
	Определить главные и второстепенные элементы. Главные элементы --
	ограничения, критерий качества, переменные. 

	Вся эта бодяга появилась в середине прошлого века. В 38 году
	появился сиплекс метод (линейное программирование) единственная
	нобелевка по экономике в СССР. Тогда американцы строили мат модель
	военных действий, чтоб организовать снабжение максимально быстро 
	(группа исследований военных операций). Задача расстановки радаров
	чтобы противник не мог подойти. Это первые 3 задачи нашего предмета.

	Рассмотрим $X$ -- некий кортеж (вектор) переменных. 
	Мы хотим найти  $X^{*}$ который является допустимым
	(удовлетворяет всем ограничениям) $X^{*} \in D$ и 
	$f(X^{*}) \le f(X)$ или $f(X^{*}) \ge f(X)$ для $\forall  X \in D$, 
	$f$ целевая функция.
	\begin{equation}
		f(X^{*},C) = \text{extr}_{X \in D} f(X,C)
	\end{equation}
	Еще более короткая формулировкаа, $C$ набор заданных параметров,
	которые не варьируются.
	\begin{equation}
		D= \{X \in \mathbb{R}^{N} \mid \phi_{j}(X,C) \le ,=,\ge b_{j},j=1 \dots M \land x_{i} \ge 0,i = 1,\dots N\}
	\end{equation}
	$\phi$ ограничения ресурсов\\
	 $M$ число ограничений,  $N$ число переменных \\
	 Выделяют отдельный класс ограничений,  $x \in [a,b]$ ,$x <,> C$
	 геометрические ограничения заменяют  $x_{i}\ge 0$. Так же
	 $M < N$ (число ограничей меньше числа  переменных)
	 \subsection{Классификация по способу принятия решений (по инф наполнению)}
	 \begin{enumerate}
	 	\item статические
		\item динамические
	 \end{enumerate}
	 Мы только статические изучаем
	 \subsection{Классификаия по типу целевой функции и по типу
	 ограничений}
	 \begin{enumerate}
	 	\item Целевая функция и ограничения выражаются 
			линейными зависимостями -- мы говорим о линейном программировании.
		\item Целевая функция квадратична, ограничения линейны -- квадратичное программирование
		\item $f(x) = f_1(x_1)f_2(x_2)\dots f_{N}(x_{N})$ -- сепарабельное программирование
		\item Если переменные целые, то программирование целочисленное
		\item Если целевая функция выпуклая, то это выпулкая задача оптимизации
	 \end{enumerate}
	 \begin{definition}[Унимодальная задача]
		 Задача у которой один экстренум
	 \end{definition}
\section{Задача безусловной оптимизации}
Есть целевая функция 
\begin{equation}
F(X) , X = (x_1,x_2,\dots,x_{N}), X \in \mathcal{D} \subseteq \mathbb{R}^{N}
\end{equation}
\begin{eqnarray}
	U_{\epsilon}(X), X \in \mathbb{R}^{N} : 0 < ||X-X'|| - ||\delta X|| = \sqrt{\sum_{i=1}^{N} (x_{i} - x_{i}')^2}  < \epsilon\\
	\delta X = (x_1 - x_1',\dots, x_{N} - x'_{N})\\
	U_{\epsilon} (X)  - \{X+ \delta X\} \text{открытый шар}
\end{eqnarray}
\textbf{
}
\begin{definition}[Точка локального экстремума]
	\label{localExtr}
\begin{equation}
	\exists  U_{\epsilon}(X^{*}) : f(X^{*}) \le  f(X) \text{~или~} f(X^{*}) \ge f(X)
\end{equation}
\end{definition}
\begin{definition} [Локальный минимум максимум]
	Значение целевой функции в точке экстремума
\end{definition}
\begin{definition}
	\label{globExtr}
	\begin{equation}
		X^{*}: \forall  x \in D: f(X^{*}) \le  f(X) \text{~или~}
		f(X^{*}) \ge  f(X)
	\end{equation}
\end{definition}
\begin{equation}
	f^{*} =  \inf_{X \in \mathcal{D}}f(X) < f(X)
\end{equation}
\begin{equation}
	f^{*} = \sup_{X \in \mathcal{D}} f(X) > f(X)
\end{equation}
\subsection{Постановка задачи оптимизации}
Пусть даны:
\begin{enumerate}
	\item функция $f$ с областью определения $\mathcal{D}_{f} \subseteq \mathbb{R}^{N}$
	\item множество $\mathcal{D} \subseteq \mathbb{R}^{N}$
\end{enumerate}
Требуется найти точку $X^{*} = (x_1^{*},x_2^{*},\dots,x^{*}_{N}) \in \mathcal{D}$ 
в которой функция достигает экстремального значения те
\begin{equation}
	X^{*}:f(X^{*}) = \text{extr}_{x \in \mathcal{D_{f}}  \cap X \in \mathcal{D}} f(X)
\end{equation}
\begin{enumerate}
	\item Задача на безусловный экстренум $\mathcal{D} = \mathcal{D_{f}} = \mathbb{R}^{N}$ 
	\item Задача на условный экстремум 
		\begin{eqnarray}
			\mathcal{D_{f}} \subset \mathbb{R}^{N}
		\end{eqnarray}
		или
		\begin{equation}
			\mathcal{D} \subset \mathbb{R}^{N}
		\end{equation}
		\begin{equation}
			\mathcal{D_{f}} \cap \mathcal{D} \neq \emptyset
		\end{equation}
\end{enumerate}
\begin{equation}
	\psi_{j}(X) \le (,=,\ge ) 0, j = 1 \dots M
\end{equation}
\begin{equation}
	\mathcal{D} = \{X \in \mathbb{R}^{N}  : 
	\psi_{j}(X) \le (,=,\ge ) ~0,j = 1 \dots M\}
\end{equation}
\begin{equation}
	\alpha_{i} \le x_{i} \le  \beta_{i} , i \in [1,N] \text{или} x_{i} \ge  0< i \in [1,N]
\end{equation}
\begin{equation}
	\mathcal{P} = \{X \in \mathbb{R}^{N} : a_i \le  x_{i} \le b_{i}
	x_{i} \ge  0,i \in [k,N]\}
\end{equation}
\begin{equation}
	\mathcal{D} = \{X \in \mathbb{R}^{N}  : 
	\psi_{j}(X) \le (,=,\ge ) ~0,j = 1 \dots M, x\in \mathcal{P}\}
\end{equation}
\begin{definition}[Избыточное ограничение]
	Избыточное ограничение является следствием других ограничений
\end{definition}
Может оказаться так что есть целевая функция и набор ограничений. Целевая
функция имеет экремум, но он может оказаться за пределами области
допустимых значений. Если экстренум на границе, то некоторые
ограничения выполняются как строгие равенства, то ресурс соответсвующий
этому ограничению называется дефицитным, ограничение активным,
если нет равенства ресурс недефицитным, ограниичение пассивное.
\begin{definition}[Стационарных точка]
	Точка в которой производной равна нулю.
\end{definition}
\begin{theorem}[Ферма, необходимое условие экстремума первого порядка]
	Пусть функция задана на вещественной оси $\mathbb{R}$,
	$x \in \mathbb{R}$  и в некоторой точке $x^{*}$ дифференцируема.
	Если в этой точке локальный экстремум, то $f'(x^{*}) = 0$
\end{theorem}
Доказательство.\\
$\delta x \ge  0$ , $f(x \pm \delta x)$ =  $f(x^{*}) \pm f'(x^{*}) \delta x + o(\delta x)$

В ряд Тейлора разложили. $x^{*}$ точка локального минимума
\[
f(x- \delta x)  -f(x^{*}) = - f'(x^{*}) \delta x \ge 0
.\] 
\[
f(x+ \delta x)  -f(x^{*}) = - f'(x^{*}) \delta x \ge 0
.\] 
\[
	\begin{cases}
		f'(x^{*}) \ge 0\\
		f'(x^{*}) \le 0
	\end{cases}
.\] 
\[
f(x^{*}) = 0
.\] 
\begin{theorem}[Ферма, многомерный случай]
	Пусть функция $f(X)$ задана на  $\mathbb{R}^{N}$. Если в точке
	$X^{*}$ локальный экстремум, то 
	\[
	f'_{x_1} = 0,\dots f'_{x_{N}} = 0
	.\] 
\end{theorem}
доказательство\\
$X^{*}$ точка локального минимума функции $f(X)$
 \[
U_{\epsilon}(X^{*})
.\] 
\[
f(X^{*}) \le  F(X)
.\] 
\begin{example}
	\[
	f(X) =  \sum_{i =1}^{N} a_i x^2_{i}
	.\] 
	\[
	f'_{x_{i}} = x_{i} = 0
	.\] 
	\[
	X^{*} = \begin{pmatrix} 
	0\\
	\dots\\
	0
	\end{pmatrix} 
	.\] 
	$X^{*}$ точка минимума , если все коэфициенты положительные,
	максимума если отрицательны
\end{example}
\begin{example}
	\[
	f(x_1,x_2) = 5x_1^2 - 6x_2^2
	.\] 
	\[
	X^{*} = (0,0)
	.\] 
	На множестве $\{x_1,0\}$ $f(x_1,0) = 5x_1^2$ возр
\end{example}
\begin{example}
	\[
	f(x_1,x_2) = (x_1-1)^{3} + (x_2 + 1)^2
	.\] 
\end{example}
\begin{theorem}[Необходимое условие экстремума второго порядка]
	Пусть функция $f(x)$ задана на  $\mathbb{R}$, и дважды непрерывно
	дифференциуема в некоторой окрестности $\mathcal{U}_{\epsilon}(x^{*})$ 
	Если в точке $x^{*}$ имеется локальный минимум (максимум),
	вторая производная бдует неотрицательная (неположительная)
\end{theorem}
Доказательство.
\[
f(x^{*} + \delta x) = f(x^{*}) + f'(x^{*}) \delta x +
\frac{1}{2} f''(x^{*}) ( \delta x )^2 + o((\delta x)^2)
.\] 
\[
f(x^{*} + \delta x) - f(x^{*}) = \frac{1}{2} f''(x^{*}) (\delta x)^2
.\] 
Если $x^{*}$ точка локального минимуму , то $f''(x^{*}) \ge 0$\\

Если $x^{*}$ точка локального максимума , то $f''(x^{*}) \le  0$
\begin{definition}[Матрица Гессе]
	\begin{equation}
		H_{ij}(X) = (f''_{x_{i} x_{j}})
	\end{equation}
\end{definition}
\begin{definition}
	Квадрантная матрица называется положительно определеннойб
	положительно полоопределенной,
	отрицательно определенной,
	отцательно полуопределенной
	\[
	Q(X) = X A X'
	.\] 
	$Q>0,Q\ge 0,<0,Q\le 0$
\end{definition}
\begin{theorem}[Критерий Сильвестра]
	\begin{enumerate}
		\item Если все угловые миноры положительны положительно определенные
		\item Кс
	\end{enumerate}
\end{theorem}
\begin{theorem}[Необходимое условие экстренума второго порядка]
	$f(X)$ задана на  $\mathbb{R}^{N}$, дважды непрерывно дифференцируема
	в некоторой области $U_{\epsilon}(X^{*})$ 
	если в точке $X^{*}$ имеет локальный минимум (максимум),
	то вычисленная в эттой точке матрица гессе неотрицательно (неположительно определена)
\end{theorem}
\begin{theorem}
	Рассмотрим функцию $f(x), x\in \mathbb{R}$ предположм, что
	$x^{*}$ стационарная точка, $f'(x^{*}) = 0$ в окрестности
	которой существует непрерывная производная
	второго порядка
	если в тчоке $x^{*}$ выполняется условие
	\begin{enumerate}
		\item $f''(x^{*}) > 0,x^{*}$ точка локального минимума
		\item $f''(x^{*}) < 0, x^{*}$ точка локального максимума
	\end{enumerate}
\end{theorem}
\begin{theorem}

\end{theorem}
\begin{definition}
	Для функции $f(X)$ определенной на  $\mathbb{R}^{N}$ 
	вектор единичной длины задает
	\begin{enumerate}
		\item направление убывания $w^{\downarrow} \in \mathbb{R}^{N}$
		\item направление возрастания $w^{\uparrow} \in \mathbb{R}^{N}$
	\end{enumerate}
	если при всех достаточно малх $\alpha > 0$ выполняется неравенство
	\begin{enumerate}
		\item 

	 \[
	f(X' + a w^{\downarrow}) < f(X')
	.\] 
\item  $f(X' + \alpha w^{\uparrow} > f(X')$
	\end{enumerate}
\end{definition}
\begin{theorem}
	Пусть фукнция $f(X)$ дифференцируема в точке  $X' \in \mathbb{R}^{N}$
	 \begin{enumerate}
	 	\item Если вектор $\vec{w} \in \mathbb{R}^{N}$ удовлетворяет 
			условию
			\[
			grad~f(X') \vec{w} < > 0
			.\] 
		то $\vec{w}$ принадлежиет множеству направний убываний
	 \end{enumerate}
\end{theorem}
\[
f(X' + \alpha \vec{w}) = f(X') + grad ~ f(X')* a \vec{w} + o(\alpha)
.\] 
\section{Классическая задача условной оптимизации}
\begin{definition}
	Задача классическая, если ограничения заданы равенством.
\end{definition}
\begin{equation}
	X^{*} : f(X^{*}) = \text{extr}_{X \in \mathcal{D}} f(X)
\end{equation}
\begin{definition}
	Вектор $v \in\mathbb{R}^{N}$ задает \textbf{возможное направление}
	в точке $X \in \mathcal{D}$ на множестве возможных значений,
	если при всех достаточно малых $\alpha > 0$ точка  $X' = X + \alpha v$ принадлежит  $\mathcal{D}$,$X' \in \mathcal{D}$


	Множество возможных направлений $\mathcal{V}_{X} \subseteq \mathbb{R}^{N}$
\end{definition}
\begin{theorem}
	Если $X^{*}$ является точкой локального минимума (максимума), то
	\begin{equation}
		\label{eq}
		\mathcal{W}^{\uparrow (\downarrow)} \cap \mathcal{V}_{X} = \emptyset
	\end{equation}
\end{theorem}
Пусть \ref{eq} неверна, $\mathcal{W}^{\uparrow} \cap \mathcal{V}_{X} \neq 0$
\[
\exists  R \in W^{\uparrow}
.\] 
\[
f(X^{*} + \alpha r) > f(X^{*}) , X^{*} + \alpha r \in \mathcal{D}
.\] 
Невозможно смещение из точки локального мин которое приводит к уменьшению целевой функции и не выходит за пределы целевой области
\begin{theorem}[Вейерштрасса]
	Пусть $\mathcal{D}$ замкунутое ограниченное множество 
	и $f(X)$ непрерывная функция. Тогда на  $\mathcal{D}$ 
	существуют точки глобального минимума и максимума
\end{theorem}
Если внутренних точек локального экстремума нет, то экстремальное
значение может достигаться только на границе области.
\begin{lemma}
	Если область допустимых 
	значений определямая системо равенств, содержит
	некоторую точку $X'$ и ее окретсность  $U_{\epsilon}(X'),X'\in \mathcal{D}, U_{\epsilon}(X') \in \mathcal{D}$, nj
	\[
	M < n
	.\] 
	При этом функции задающие ограничения дифференцируемаые
\end{lemma}
\[
	\psi(X' + \delta X) = \psi_{j} (X') + \text{grad} \psi_{j}(X') \delta X + o(||\delta X||)
.\] 
\[
\psi(X' + \delta X) = \psi(X') = 0
.\] 
\[
	\text{grad} \psi_{j} * \delta X = 0, j = 1 \dots M
.\] 
Это линейная система уравнений относительно $\delta X$ 
Это меньще $N < M$
\begin{lemma}
	Предположим, что область допустимых значений, содержит
	хотя бы одну точку, если $M < N$ и якобиан  $J(X')$ 
	составленный из функций  $\psi_{J}(X)$ имеет  в этой точке 
	ранг, равный $M$ то область Допустимых значений 
	 $\mathcal{D}$ вместе со своей точкой $X'$ содержит некоторую 
	 непустую окрестность
\end{lemma}
\[
	\psi(X' + \delta X) = grad~\psi_{j}(X') * \delta X + o(||\delta X||)
.\] 
В классической задаче оптимизации число ограничений строго меньше
числа переменных
\begin{definition}
	Допустимой окрестностью $U_{\epsilon}^{D}(X)$ 
	точки $X$ называется ее окрестность целиком содержащаяся в допустимом множестве
	 $\mathcal{D}$
\end{definition}
\section{Метод множителей Лагранжа}
\begin{definition}
	\begin{equation}
		L(\Lambda,\lambda_0,X) = \lambda_0 f(X) +
		\sum_{j = 1}^{M} \lambda_{j} \psi_{j}(X)
	\end{equation}
	\begin{equation}
		L(\Lambda,\lambda_0,X) = \lambda_0 f(X) - \sum_{j}^{M} \lambda_{j} \psi_{j}(X)
	\end{equation}
\end{definition}
\begin{theorem}[правило множителей Лагранжа]
	Пусть в окрестности $U_{\epsilon}(X^{*}) \subset \mathcal{D}$ 
	точки $X^{*} \in \mathbb{R}^{N}$ 
	\begin{enumerate}
		\item функции $f(X)$ ,  $\psi(X)$ , $j = 1 \dots M$ 
			непрерывно диццеренцируемы
		\item hранг матрцы Якоби в точке  $X^{*}$ равен $M$
	\end{enumerate}
	Если $X^{*}$ точка локального оптимума задачи то 
	существуют такие неравные одновремено нулю
	вектор $\Lambda^{*}$ и параметр $\lambda^{*}_{0}$ 
	что точка $(\Lambda^{*},j_1,X^{*}$ является стационарной точки задачи на
	безусловный экстремум
\end{theorem}
Пусть $X^{*} \in \mathcal{D}$ точка эстремума задачи
$X^{*}$ стационарная точка Функции Лагранжа, все частные производные равны 0
Нужно рассмотреть 2 случая $\lambda_0$ = 0, $\lambda_0 = 1$	,
$\lambda_0 \neq 0$ система содержит $N+M$ неизвестных,  $\lambda_0 = 0$ $\Lambda=0$ задачи не имеет смысла

Найдем вторую производную функции Лагранжа
\begin{equation}
	H(\Lambda,X) =
	\begin{pmatrix} 
		0 & J(X) \\
		J^{T}(X) & L''_{XX}(\Lambda,X)
	\end{pmatrix} 
\end{equation}
\begin{theorem}[Необходимое услове экстремума второго порядка]
	Пусть в окрестности $U_{\epsilon}(X^{*}) \subset \mathcal{D}$ 
	\begin{enumerate}
		\item функции $f(X), \psi_{j}(X), j = 1 \dots M$ дважды
			непрерывно дифференцируемы
		\item Ранг матрицы Якоби $J$ равен  $M$
	\end{enumerate}
	Тогда $X^{*}$
\end{theorem}
\section{Интерпретаця множителей Лагранжа}
\begin{equation}
X^{*} : f(X^{*})  = \extr_{X \in D} f(X)
\end{equation} 
\begin{equation}
	D = \{X \in \mathbb{R}^{N} \mid \phi_{j}(X)  = b_{j}\}
\end{equation} 
\begin{equation}
L(\Lambda,X) = f(X) - \sum_{j = 1}^{M} \lambda_{j} (\phi_{j} (X) - b_{j})
\end{equation} 
\begin{equation}
\frac{\partial L(\Lambda,X)}{\partial x_{i}} =
\frac{\partial f(X)}{\partial x_{i}} - \sum_{j=1}^{M} \lambda_{j} \frac{\partial \phi_{j}(X)}{\partial _{i} } = 0
\end{equation} 
Пусть $X^{*}=  (x_1^{*},\dots x^{*}_{N)}$ 
является точкой условного экстремума $f(X)$
 $f(X^{*}0,\phi_{j}(X^{*})$ рассматриваем как функции
 параметров $b_{j}$
  \begin{equation}
  x_{i}^{*} = x_1(b_1,\dots,b_{M})
  \end{equation} 
  \begin{equation}
  f(X^{*}) = f(X^{*}(b_1 \dots b_{M}))
  \end{equation} 
  \begin{equation}
  \phi_{j}(X^{*}) = \phi_{j}(X^{*}(b_1 \dots b_{M}))
  \end{equation} 
  \begin{equation}
  \bm{ \frac{\partial f}{\partial b_{k}} }_{X^{*}} = 
  \sum_{i = 1}^{N} \bm{ \frac{\partial f}{\partial x_{i}} }_{X^{*}} \frac{d x_1}{x_{k}}
  \end{equation} 

  \subsection{Оценка точности апроксимации}
  тут рассматриваем точность, из-за разложения в ряд Тейлора

  Оценим точность апроксимации $\delta f(X)$
  \subsection{Анализ чувствительности методом Якоби}
  \begin{equation}
	  D = \{X \in \mathbb{R}^{N} \mid \phi_{J}(X) = b_{j},
	  j = 1 \dots M\} \subset \mathbb{R}^{N}
  \end{equation} 
  \begin{equation}
  \delta B = (\delta b_1, \dots \delta b_{M} )
  \end{equation} 
  \begin{equation}
  \rank \mathbf{J}(X) = M
  \end{equation} 
  \begin{equation}
  \delta X = ( \delta S, \delta Z)
  \end{equation} 
  \begin{equation}
  \delta f(X)  = \grad_{S} f(X) \delta S^{T} +
  \grad_{Z} f(X) \delta Z^{T}
  \end{equation} 
  \begin{equation}
  \delta \phi_{j} (X) = \delta b_{j}
  \end{equation} 
  \begin{equation}
  \mathbf{J}(X) \delta S^{T} + \mathbf{C} \delta S^{T}
  \end{equation} 
  \begin{equation}
  \delta f(X) = \grad_{S} f(X) \mathbf{J}^{-1}_{0} (X) \delta B^{T}  + \grad_{(*)} f(X) \Delta Z^{T}
  \end{equation} 
  \begin{equation}
  \delta f(X^{*}) = \grad_{S} f(X^{*}) \mathbf{J}_{0}^{-1}(X^{*}) \delta B^{T}
  \end{equation} 
  \begin{equation}
	  \frac{\delta f(X^{*})}{\delta B^{T}} = \grad_{S} f(X^{*}) \mathbf{J}_{0}^{-1}(X^{*})
  \end{equation} 
  \section{Линейное программирование}
  Целевая функция линейна, ограничения линейные равенства или неравенства.
  \begin{example}[Задача о пищевом рационе]
	  Есть набор питания $P_1,\dots P_{N}$ ,
	  стоимостью $c_1 \dots c_{N}$ 
	  за единицу продукта. Пусть
	  весовая едтгтца каждого продукта
	  содержит $a_{i1}$ белка, $a_{i 2}$ жиров,
	  $a_{i 3}$ углеводов, $a_{i 4}, a_{i 5}, a_{i 6}$ 
	  Нужно приобрести такое количесвто продуктов, чтоб
	  их стоимость была минимальна, а составленый рацион соедержал $b_{j} j = 1 \dots 6$ жиров и прочего (гдето равенства, где то $\le $)

	  Это можно записать в виде таблицы

	  \begin{equation}
	  X = (x_1 \dots x_{n})
	  \end{equation} 
	  Количества продуктов
	  \begin{equation}
	  f(X) = \sum_{i= 1}^{N} c_{i} x_{i} \to \min
	  \end{equation} 
	  Целевая функция
	  \begin{equation}
	  \sum_{i =1}^{N} a_{i 1} x_{i} \ge  b_1
	  \end{equation} 
	  Задали ограничения .  Естественные ограничения $x_{i}\ge 0, i =1 \dots N$
  \end{example}
  \begin{example}[Распределение ресурсов]
		\begin{enumerate}
			\item $P_1,P_2,\dots,P_{N}$ 
				продукция
			\item $R_1,R_2,\dots R_{M}$ сырье в количестве 
				$b_1,b_2,\dots b_{M}$ 
			\item стоимость едницы ресурсов $d_1 \dots d_{M}$
		\end{enumerate}
		Сколько надо выпустить продукции чтобы получить
		максимум прибыль
		\begin{equation}
		f(X) = \sum_{i =1}^{N} c_{i} x_{i} -
		\sum_{i = 1}^{N} \left( c_{i} - \sum_{j=1}^{M} a_{ji} dj\right) x_{i} \to \max
		\end{equation} 
  \end{example}
  \begin{example}[Загрузка оборудования]
	  Есть три типа станков $N_1,N_2,N_3$
	  станки производят 6 видов продукции $P_{j},j=1 \dots 6$ . $P_{j}$ приносит доод $c_{j}$. 
	  План $\ge   b_{j}$ продукции $P_{j}$
  \end{example}
  \begin{example}
  	Целевое программирование
	\begin{enumerate}
		\item На $M$ станках изоогатавливают  $N$ видов 
			изделий  $P_{i}$
		 \item Прибыль в расчете на одно изделие $c_{i}$
		 \item станок можем использовать по
			 5 часов. Если больше то продляем на $\Delta t$ часов.
		\item  $d_{j}$ цена сверхурочных работ в час
	\end{enumerate}
  \end{example}
\end{document}
 
